\documentclass{article}
\usepackage[nonatbib]{neurips_data_2021}
\usepackage[sort&compress,numbers]{natbib}
\usepackage[utf8]{inputenc}
\usepackage{booktabs}
\usepackage{hyperref}
\usepackage[capitalize]{cleveref}
\usepackage{graphicx}
\usepackage{xcolor}
\usepackage{subcaption}
\newcommand{\tbo}[1]{\textcolor{teal}{/* TBo: #1 */}}
\title{The CPD Data Set: Personnel, Use of Force, and Complaints in the Chicago Police Department}

\begin{document}


\maketitle

\begin{abstract}
The lack of accessibility to data on policing has severely limited researchers’
ability to conduct thorough quantitative analyses on police activity and
behavior, particularly with regard to predicting and explaining police
violence. In the present work, we provide a new dataset that contains
information on the personnel, activities, use of force, and complaints in the
Chicago Police Department (CPD). The raw data, obtained from the CPD via a
series of requests under the Freedom of Information Act (FOIA), consists of 35
unlinked, inconsistent, and undocumented spreadsheets. Our paper provides a
cleaned, linked, and documented version of this data that can be reproducibly
generated via open source code. We provide a detailed description of the
dataset contents, the procedures for cleaning the data, and summary statistics.
The data have a rich variety of uses, such as prediction (e.g., predicting
misconduct from officer traits, experience, and assigned units), network
analysis (e.g., detecting communities within the social network of officers
co-listed on complaints), spatiotemporal data analysis (e.g., investigating
patterns of officer shooting events), causal inference (e.g., tracking the
effects of new disciplinary practices, new training techniques, and new
oversight on complaints and use of force), and much more. Access to this
dataset will enable the machine learning community to meaningfully engage with
the problem of police violence.
\end{abstract}

\textbf{Repository: \url{https://github.com/chicago-police-violence/data}}\\
\textbf{Instructions:} run \url{make} in the repository root directory. See \url{README.md} for detailed instructions.\\
\textbf{Data Set:} may be found in the \url{final/} directory after running \url{make}.

% !TEX root = ../main.tex


\section{Introduction} \label{sec:intro}

%%The lack of accessibility to data on policing has severely limited researchers’ ability to conduct thorough quantitative analyses on police activity and behavior, particularly with regard to predicting and explaining police violence. 


Over the last few decades, the problem of police brutality has emerged as a mainstream social and political issue. In the United States,  policymakers have identified the lack of data collection and scarcity of reliable downstream data analysis of policing activities as key elements of this problem. Indeed, improving data collection, and related analysis, comes with the promise of enhancing police, as well as promoting effective police reforms. For example, the 21st Century Task Force, a presidential initiative proposed in 2014 to build trust between citizens and their law enforcement officers, identified the lack of policing data as a major contributor to the inability of communities and law enforcement agencies to ``make informed policy and practice adjustments based on good information''. Relatedly, the task force highlighted the need to collect more and better data, to ``improve the level of trust, transparency, and accountability between communities and law enforcement agencies.''\footnote{\url{https://cops.usdoj.gov/RIC/Publications/cops-p341-pub.pdf}}. 

Small steps in this directions are underway: the Department of Justice (DOJ) is currently collecting data on arrest-related deaths through the implementation of the ``Death in Custody Reporting Act'' of 2013 (DCRA, P.L. 113-242). Under this act, states are required to provide data regarding the death of any person related to policing activity (detained, under arrest, in the process of being arrested, etc.). Starting January 2019, also the Federal Bureau of Investigation(FBI) has started collecting data within the ``National Use-of-Force Data Collection'' program.\footnote{\url{https://crime-data-explorer.app.cloud.gov/pages/home}}. This program is open to all federal, state, local, and tribal law enforcement and investigative agencies, although participation to it is not enforced, rather on voluntary basis. 

However, to the day,  policymakers are fundamentally limited by the insufficient amount, quality and accessibility of policing data. And, more generally, the data-scarcity has prevented the wider scientific community from conducting thorough quantitative analysis on police activity and behavior (see, e.g.\ \citet{peeples2019data,peeples2020data} for recent overviews of the problem of the lack of data in the context of police brutality). As a consequence, we are still far from having developed effective ``data-driven'' policy-making pipelines. Indeed, recent work has investigated how the incompleteness and bias of available data can lead to unwarranted data-driven solutions: \citet{richardson2019dirty} discuss how developing predictive policing system which leverage ``bad'' historical data --- corrupted incomplete and racially biased --- can produce dangerous downstream predictions, potentially violating individual civil rights. Relatedly, machine learning scientists have started investigating the problem of ``algorithmic fairness'', analyzing  risks and flaws related to the employment of black-box, automated decision systems within complex data-driven policy making procedures \citep{veale2018fairness, sloane2019ai, d2020fairness}. Because of all these reasons, it is imperative for the community to collect, curate and provide freely-accessible, easy-to-use data on policing activity.


In the present work, we provide a new dataset that contains information on the personnel, activities, use of force, and complaints in the Chicago Police Department (CPD). \textcolor{red}{maybe short discussion of why Chicago is ``special''? + cite some work on policing activity in Chicago? + cite some relevant recent work eg \citep{ba2021role}}. The original data comprises a collection of datasets, each obtained following different requests covered by the Freedom of Information Act (FOIA) to the Chicago Police Department (CPD) and the Civilian Office of Police Accountability (COPA) on the CPD's personnel and its activities. The data on complaints against police include complaints filed by citizens or internally by other members of the department.  Prior to our work, joint, coherent use of these different dataset for data-analysis purposes was highly non-trivial: for example, publicly available data does not contain coherent unique identifiers for officers across the different datasets. We discuss in detail in \Cref{sec:data} our data cleaning, linking and formatting process. Next, we provide examples and use-cases on the datasets in \Cref{sec:analysis}. These include, but are not limited to, prediction tasks (e.g., predicting officer's misconduct on the basis of their traits, experience, and assigned units), network analysis (e.g., detecting communities within the social network of officers co-listed on complaints), spatiotemporal data analysis (e.g., investigating patterns of officer shooting events), causal inference (e.g., tracking the effects of new disciplinary practices, new training techniques, and new oversight on complaints and use of force). We conclude with a discussion of intended use, and future research directions in \Cref{sec:discussion}.


%%%%%% DARIA'S TEXT

%
%There has been some ML work done on individual correlates that are associated
%with (and potentially predictive of) police misconduct. Scholars have looked at
%the association between misconduct and an individual officer's race, gender,
%years on the force, prior instances of misconduct, stages of officer career,
%assigned unit. But research suggests that violence emerges not just at the
%individual level but also at the population level from the interaction of
%individuals in groups—think of gangs. 
%
% 
%We offer a dataset that allows ML to investigate the influence of police social
%networks on police misconduct. The dataset allows ML to explore features of
%police social networks that might be associated with and potentially predictive
%of adverse police encounters with civilians or other misconduct. These network
%correlates can include the architecture and structures of social networks in
%departments and smaller subunits within a department—for example, the number of
%connections each officer has, the strength of those connections, the bridge
%positions that some officers have across subunits. More innovatively, ML can be
%used in the way it is often used to study disease dynamics: ML can find the
%patterns of behavior (clustering, outbreaks, spreading, geographic hotspots) on
%a network that might be associated with violence. 
%
% 
%We show how to use complaint datasets from a police department in a major city
%to investigate the influence of social networks on police misconduct and
%violence. 
%
%
%The original data was obtained following a series of requests covered by the
%Freedom of Information Act (FOIA) to the Chicago Police Department (CPD) and
%the Civilian Office of Police Accountability (COPA). The information which was
%requested pertained to CPD's personnel and its activities.
%
%\tbo{merge}
%The data on complaints against police include complaints filed by citizens or internally by other members of the department. This data was obtained by Jamie Kalven, an independent journalist represented by the University of Chicago’s Mandel Legal Aid Clinic, who filed Illinois Freedom of Information Act requests with the Chicago Police Department. These FOIA requests asked for documents containing the names of repeat police complainees, lists that had been produced in discovery in lawsuits alleging police abuse. In Kalven v. City of Chicago, an Illinois appellate court issued a general ruling in March of 2014 that documents bearing on allegations of police abuse are public information. Following the decision, the non-profit Invisible Institute began to collaborate with Kalven and the Mandel Legal Aid Clinic to follow up on earlier FOIA requests and to file new ones. The data disclosed in response to these earlier and now ongoing FOIA requests are uploaded on the Invisible Institute’s website and are made publicly available. 
%
%\textbf{TODO:} ideally say a bit more about the history of these FOIA requests.
%Apparently they were initiated by individual journalists and lawyers and were
%later coordinated by the Invisible Institute which ultimately became the
%central location were (almost) all the data is currently available.
%
%\begin{table}[h]
%	\begin{center}
%\begin{tabular}{@{}llll@{}}
%	\toprule
%	request \#&received&requested&description\\
%\midrule
%	\texttt{P0-58155}&2017-04-17& &Officer roster\\
%	\texttt{P4-41436}&2018-03-21& &Officer roster\\
%		\texttt{P0-52262}&2016-12-04&2016-09-19&Unit assignment\\
%		\texttt{16-1105}&2016-03-11&2016-02-10&Unit assignment\\
%	\texttt{P0-46957}&2016-06-29&2016-04-22&Complaints (CPD)\\
%	\texttt{18-060-425}&2018-08-28&2018-08-20&Complaints (COPA)\\
%	\texttt{P0-46360}& & &Tactical Response Reports\\
%\bottomrule
%\end{tabular}
%\caption{Summary of the FOIA requests to the CPD and COPA contained in our repository.}
%\label{table:summary}
%\end{center}
%\end{table}
%
%Each FOIA request is identified by a request number, \cref{table:summary} gives
%an overview of all the requests made to the CPD and COPA that are present in
%our repository. This information is also available in the file
%\texttt{dataset.csv} in the root folder of the repository. The original
%data—that is the files received after each FOIA request—are present in the
%\texttt{raw/} folder of the repository, with one subfolder for each request,
%identified by the request number. When available, each subfolder also contains
%the formal request letter as well as the reply letter from the CPD or COPA,
%which are useful in understanding what data was included in each dataset.
%Some additional comments about the data:
%\begin{itemize}
%	\item \emph{Officer roster:} lists all officers (past or present) employed
%		by the CPD along with attributes such as year of birth, age, race,
%		gender, appointment date, resignation date, etc.
%	\item \emph{Unit assignment:} the CPD is organized into (500 or so?) units.
%		Each officer can be assigned to one or multiple units and these
%		assignments can change over time. The unit assignment datasets contain
%		one record for each officer and each unit they were assigned to, with
%		the start date and end date of this assignment.
%	\item \emph{Complaints:} formal complaints filed by citizens against police
%		officers. Complaints are identified by a complaint number, and there is
%		one record for each complaint and each officer listed on the complaint,
%		indicating the allegation made against them, result of the
%		investigation of the allegation (with possible sanction), etc.
%	\item \emph{Tactical Response Reports:} these are forms that officers are
%		required to file after each incident for which the officer's response
%		involved use of force.
%\end{itemize}
%
%Let us already mention an inherent difficulty in making use of this data, which
%will be discussed in \cref{sec:linking}: there is no number/identifier which
%uniquely identifies officers across datasets. Such an identifier probably
%exists internally in the CPD, but was never included in the data released to
%the public. One would be tempted to believe that the \emph{badge number} (also
%sometimes referred to as \emph{star number} of an officer is such an
%identifier, but unfortunately, it changes over the course of an officer's
%career in the CPD, and a given badge number can be reassigned to different
%officers when they are no longer in use.
%
%
%\subsection{Related Work}
%
%Police records datasets
%
%This data was originally obtained by the Invisible Institute and has been publicly available for about 3 years here \url{https://github.com/invinst/chicago-police-data}
%(limitations: not well organized or reproducible, problematic/incorrect linkage, missing unit information, units are not semantically grouped)
%
%Gang violence data



\section{The CPD Data} \label{sec:data}
The original raw data released by the CPD, 
as well as the code to clean the data and generate this document, 
are both available at \url{https://github.com/chicago-police-violence/data}.

\subsection{Data origin}
This data was obtained by Jamie Kalven, an independent journalist represented
by the University of Chicago’s Mandel Legal Aid Clinic, who filed Illinois
Freedom of Information Act requests with the Chicago Police Department. These
FOIA requests asked for documents containing the names of repeat police
complainees, lists that had been produced in discovery in lawsuits alleging
police abuse. In Kalven v. City of Chicago, an Illinois appellate court issued
a general ruling in March of 2014 that documents bearing on allegations of
police abuse are public information. Following the decision, the non-profit
Invisible Institute began to collaborate with Kalven and the Mandel Legal Aid
Clinic to follow up on earlier FOIA requests and to file new ones. The data
disclosed in response to these earlier and now ongoing FOIA requests are
uploaded on the Invisible Institute’s website and are made publicly available. 

\textbf{TODO:} ideally say a bit more about the history of these FOIA requests.
Apparently they were initiated by individual journalists and lawyers and were
later coordinated by the Invisible Institute which ultimately became the
central location were (almost) all the data is currently available.




\begin{table}[h]
	\begin{center}
\begin{tabular}{@{}lllll@{}}
	\toprule
	request \#&received&requested&description\\
\midrule
	\texttt{P0-58155}&2017-04-17& &Officer roster\\
	\texttt{P4-41436}&2018-03-21& &Officer roster\\
	\texttt{P0-52262}&2016-12-04&2016-09-19&Unit assignment\\
	\texttt{16-1105}&2016-03-11&2016-02-10&Unit assignment\\
	\texttt{P0-46957}&2016-06-29&2016-04-22&Complaints (CPD)\\
	\texttt{18-060-425}&2018-08-28&2018-08-20&Complaints (COPA)\\
	\texttt{P0-46360}& & &Tactical Response Reports\\
	\texttt{P0-46987}&2016-05-13&2016-04-25&Unit names\\
	\texttt{P0-61715}& &2017-07-26&Awards\\
	\texttt{P5-06887}&2019-10-11&2019-07-19&Awards\\
	&2017-09-27&2017-09-13&Salary\\
\bottomrule
\end{tabular}
\caption{Summary of the FOIA requests contained in our repository (blanks are missing entries).}
\label{table:summary}
\end{center}
\end{table}



Each FOIA request is identified by a request number, \cref{table:summary} gives
an overview of all the requests made to the CPD and COPA that are present in
our repository. This information is also available in the file
\texttt{dataset.csv} in the root folder of the repository. The original
data—that is the files received after each FOIA request—are present in the
\texttt{raw/} folder of the repository, with one subfolder for each request,
identified by the request number. When available, each subfolder also contains
the formal request letter as well as the reply letter from the CPD or COPA,
which are useful in understanding what data was included in each dataset.
Some additional comments about the data:
\begin{itemize}
	\item \emph{Officer roster:} lists all officers (past or present) employed
		by the CPD along with attributes such as year of birth, age, race,
		gender, appointment date, resignation date, etc.
	\item \emph{Unit assignment:} the CPD is organized into (500 or so?) units.
		Each officer can be assigned to one or multiple units and these
		assignments can change over time. The unit assignment datasets contain
		one record for each officer and each unit they were assigned to, with
		the start date and end date of this assignment.
	\item \emph{Complaints:} formal complaints filed by citizens against police
		officers. Complaints are identified by a complaint number, and there is
		one record for each complaint and each officer listed on the complaint,
		indicating the allegation made against them, result of the
		investigation of the allegation (with possible sanction), etc.
	\item \emph{Tactical Response Reports:} these are forms that officers are
		required to file after each incident for which the officer's response
		involved use of force.
\end{itemize}


\paragraph{Technical choices.} Code is written in Python, use of \texttt{Make}
to coordinate the various processing steps and easily reproduce them.
\textbf{TODO:} say a bit more about the requirements of the environment, or
maybe simply refer to the README?



\subsection{Initial Cleaning}

The files initially released by the CPD as a reply to the FOIA requests are for
the most part Excel spreadsheets, with inconsistent formatting and which can
thus be difficult to process programmatically. As an example, the reader is
invited to open the file
\texttt{p046957\_-\_report\_1.1\_-\_all\_complaints\_in\_time\_frame.xls}
available in the folder \texttt{raw/P0-46957/}. As can be seen, each record in
this file is spread over two rows of the spreadsheet, with the field names
repeated at the beginning of the second row for each record.

The goal of the cleaning step is thus to produce ``reasonable'' CSV files from
the original files, with the minimum requirement that each record be presented
on a single line after this step. The code for this cleaning step is contained
in the files \texttt{datasets.py}, \texttt{parse.py} and
\texttt{parse\_p046957.py} in the \texttt{src/} folder and the entire step can
be applied by running \texttt{make parse} in the root of the repository. This
creates the folder \texttt{parsed/} containing the clean CSV files.

As can be seen by inspecting the code, the decisions made at this stage are, we
believe, uncontroversial as they only consists of:
\begin{itemize}
	\item unifying field names across datasets, so that the same type of data
		is always identified in the saw way (for example, \texttt{Appointment
		Date}, \texttt{Appt Date}, \texttt{appointment\_date} are all mapped to
		\texttt{appointment\_date}.
	\item unifying field values across datasets. For example, the gender of an
		officer is indicated as a single letter \texttt{M} or \texttt{F} in
		some datasets or as \texttt{Male}, \texttt{Female} (based on the data
		release, it does not seem that the system used by the CPD has an option
		to represent non-binary officers).
	\item parsing values into the correct data type or format. For example,
		dates are formatted differently depending on the dataset, and we map
		everything to the ISO 8601 format.
\end{itemize}

Consequently, for someone planning to use the data in the present repository,
there is virtually no reason not to start at the minimum from the output of
this cleaning step. The subsequent steps required making more difficult and
debatable decisions, so depending on the application, researchers might want to
perform them differently, but in all cases, those alternative decisions can
branch off from the output of the cleaning step.

\subsection{Linking and Merging Datasets}\label{sec:linking}

\subsubsection{Officer matching}

As already alluded to, the main challenge at this step is that there is no
identifier uniquely identifying officers across records. In other words, there
is no foolproof way to know if two different records correspond to the same
officer, \emph{even within the same dataset} (for example the same officer
could appear with slightly different attributes on two different complaint
records).

We thus need to design a matching method striking a balance between
\begin{itemize}
	\item being loose enough to avoid type II error (false negatives). If the
		same officer appears with slightly different attributes across two
		records, we do not want our matching method to believe it is two
		different officers.
	\item being strict enough to avoid type I error (false positives). We do
		not want to merge two different officers into a single identity.
\end{itemize}

The difficulty in achieving this balance is that perhaps surprisingly
\emph{none of the attributes of a given police officer are guaranteed to be
stable over time}. Most notably, officers' names change over time, for example
to fix data entry errors or in case of legal name changes. However, we observed
two attributes, present in almost all original datasets and which seem
remarkably stable over the time: \emph{appointment date} and \emph{birthyear}.
These two attributes thus proved very valuable to disambiguate officers with
identical names.

In order to match officers across two datasets, we developed an \emph{iterative
pairwise matching procedure} that makes iterative passes over the datasets.
During each pass, a subset of the officer attributes present in both datasets
is selected as the matching criterion, and a pair of officers (one from each
dataset) is identified as a \emph{match} if (i) their attributes from the
chosen subset match, and (ii) if they are the only two officers matching on
these attributes. Once a pair is identified as a match, it is put aside, and
the next pass is performed on the remaining unmatched officers. After all the
passes are done the leftover officers are declared as different officers. By
constructing a hash table mapping a subset of attributes to the list of
officers sharing these attributes, each pass can be performed in linear time,
so the overall running time of the procedure is $O\big(P(N_1+N_2)\big)$ where
$P$ is the number of passes and $N_1, N_2$ are the number of officers in each
dataset.

To fully specify the matching procedure we thus need to specify which subset of
attributes is chosen as the matching criterion at each pass. For this, we go
from the stricter to the looser criterion: for the first pass, we choose
a subset \emph{all} the attributes which are present in two datasets, and then
start removing attributes one by one. For example, one can remove the
\emph{last name} attribute for the second pass to match a pair of officers
whose last names are different but match on all the remaining attributes, thus
identifying an officer whose last name changed between the releases of the two
datasets. The advantage of going from stricter to looser is two fold:
\begin{itemize}
	\item starting from the strictest set of attributes identifies the
		\emph{unambiguously matching pairs}, that is, the ones which are
		a clear match and which, thankfully, constitute the vast majority of
		officers (typically around 80-90\% of officers are matched during the
		first pass \textbf{TODO check}). Since the next passes will only be
		performed on the remaining unmatched officers, this removes a lot of
		potential ambiguities which could occur once the set of attributes is
		reduced.
	\item since the vast majority of officers is matched during the first pass,
		it becomes feasible during the subsequent passes to visually inspect all
		the matched pairs and assess whether the chosen set of attributes was
		too strict or too lose (\textbf{TODO:} explain how to activate
		debugging information in the code).
\end{itemize}

We note that there is still some amount of subjective judgment involved,
following a visual inspection of the second and subsequent passes, to decide
which sets attributes are ``acceptable'' (that is, for which the probability of
two persons sharing these attributes in a population of the size of the CPD is
extremely small). This is also how we decide that sufficiently many passes have
been performed: when it would seem likely to introduce a type I error by
matching any of the remaining officers. As a general rule of thumb we erred on
the side of favoring type II errors over type I errors. That is, we only
matched officers when it would seem extremely unlikely that they correspond to
two different individuals).

\textbf{TODO:} table of officer attributes in each dataset

\textbf{TODO:} explain the few subtleties where we don't use equality to match
attributes (for example when matching age and birthyear, where the age only
lets us identify the birthyear with an accuracy of 1 year). Or with stars,
where we test whether a star number is contained in the subset of known stars
for this officer.

With this procedure at hand we can thus link officers across datasets starting
from the most similar datasets first (for which we expect to have the least
amount of ambiguity). That is we first link \texttt{P0-58155} to
\texttt{P4-41436}, then \texttt{P0-52262} to \texttt{16-1105} and then the
remaining datasets \textbf{expand}

\textbf{TODO} give in appendix table summarizing which subset of attributes are
used at each linking operation 

\subsubsection{Roster consolidation}

A byproduct of matching officers across datasets is that for each officer, we
now have as many “profiles” as the number of datasets in which they appear,
where by \emph{profile} we mean a collection of attributes. Note that each
profile can contain a different subset of attributes (since not all attributes
are present in each dataset) and that a given attribute might take a different
value in different profiles of the same officer (since the iterative matching
procedure is not restricted to performing strict matching).

We are thus faced with the task of “consolidating” the different profiles of
a given officer into a single profile. Of course, if an attribute is present in
a single profile, and absent from the others, this is the value we keep in the
consolidated profile. But if an attribute appears with different values across
different profiles, we choose the value coming from the profile corresponding
to the \emph{most recent data release}.

\subsubsection{Unit assignment history} TODO.


\subsection{Unit identification and binning}


% !TEX root = ../main.tex

\section{Examples and data analysis} \label{sec:analysis}

In this section, we provide examples and use-cases on the dataset. Code to reproduce our examples is available online, in the \texttt{examples} folder.

\subsection{Summary statistics}

First, we start by providing some summary statistics. 

\subsubsection{The roster file}
The file \texttt{roster.csv} contains information about $N=35{,}430$ CPD officers. For each officer --- identified by a Unique Identifier (``uid''), \texttt{roster.csv} provides, among other covariates, the officer's name, gender, race, birthyear, appointment and resignation dates. It is straightforward  to use the data to generate summary statistics, e.g., using appointment and resignation dates to understand the number of appointments, retirements as well as active officers over the years of  (\Cref{fig:history}). We report in \Cref{tab:stats} some 
\begin{figure}[h] 
	\includegraphics[width=\textwidth]{figs/history} 
	\caption{Historical data from the CPD. The left subplot shows the number of officers' appointments, the central subplot the number of officers' resignations, while the right subplot shows the number of active officers in the database (vertical axis) as a function of time (horizontal axis).} \label{fig:history}
\end{figure}

\begin{table}[h]
\begin{tabular}{l|c|c|c|c|c|c|c|c|c|}
\cline{2-3} \cline{5-10}
                                               & \multicolumn{2}{c|}{\textit{\textbf{Gender}}} & \multicolumn{1}{l|}{} & \multicolumn{6}{c|}{\textit{\textbf{CPD Race Category}}}                                                                                                                                                   \\ \cline{2-3} \cline{5-10} 
                                               & \textit{\textbf{M}}   & \textit{\textbf{F}}   &                       & \textit{\textbf{White}} & \textit{\textbf{Black}} & \multicolumn{1}{l|}{\textit{\textbf{Hisp.}}} & \textit{\textbf{Asian}} & \multicolumn{1}{l|}{\textit{\textbf{Am. Ind.}}} & \textit{\textbf{Bl. Hisp.}} \\ \cline{1-3} \cline{5-10} 
\multicolumn{1}{|c|}{\textit{\textbf{All}}}    & 28316                 & 7122                  &                       & 21047                   & 8599                    & 4811                                         & 582                     & 67                                              & 9                           \\ \cline{1-3} \cline{5-10} 
\multicolumn{1}{|c|}{\textit{\textbf{Active}}} & 11118                 & 4452                  &                       & 7241                    & 3895                    & 3596                                         & 467                     & 40                                              & 9                           \\ \cline{1-3} \cline{5-10} 
\end{tabular}
\caption{Summary statistics (gender, race) for all officer (first row), as well as for active officers --- those whose resignation date is not before Jan 1st 2019.} \label{tab:stats}
\end{table}

\begin{figure}[h] 
	\includegraphics[width=\textwidth]{figs/history_by} 
	\caption{Historical data from the CPD. Birthyears for officers in the CPD dataset: blue dots count all officers, while red count only officers who are ``active'' as of January 1st, 2019.} \label{fig:history_by}
\end{figure}

\subsection{Tactical response reports}

The file \texttt{tactical\_response\_reports.csv} contains information about tactical response reports, filed as a consequence of  events involving use of force of CPD officers. This dataset contains $9{,}246$ distinct events, as identified by the index ``\texttt{event\_no}'', between January 17th 2004 and April 12th 2016. For each event, the time and location is recorded. Details about the officers involved, as well as the civilian subject are available too. 

\begin{figure}[h] 
	\includegraphics[width=\textwidth]{figs/trrs_times} 
	\caption{Temporal data for TRRs.} \label{fig:trrs_times}
\end{figure}

\begin{figure}[h] 
	\includegraphics[width=\textwidth]{figs/trr_stats} 
	\caption{Temporal data for TRRs.} \label{fig:trrs_stats1}
\end{figure}

\begin{figure}[h] 
	\includegraphics[width=\textwidth]{figs/trr_stats_race_race} 
	\caption{Temporal data for TRRs.} \label{fig:trrs_stats2}
\end{figure}

\subsection{Complaints}

\begin{figure}[h] 
	\includegraphics[width=\textwidth]{figs/complaints_times} 
	\caption{Temporal data for complaints.} \label{fig:complaints}
\end{figure}
\subsection{Salary}

\begin{figure}[h] 
\includegraphics[width=\textwidth]{figs/salary} 
\caption{Historical data from the CPD. Salary versus experience in each
position, years x axis, salary (USD) y axis, for a few of the most common
positions. Lines are means, bars indicate 1 std dev above and below.} \label{fig:salary}
\end{figure}

\begin{figure}[h] 
\includegraphics[width=\textwidth]{figs/position_race} 
\caption{Historical data from the CPD. Fraction of officers in representative positions 
per race category.} \label{fig:salary}
\end{figure}

\begin{figure}[h] 
\includegraphics[width=\textwidth]{figs/position_gender} 
\caption{Historical data from the CPD. Fraction of officers in representative positions 
per gender category.} \label{fig:salary}
\end{figure}

\subsection{Awards}

\begin{figure}[h] 
\includegraphics[width=\textwidth]{figs/awards} 
\caption{Historical data from the CPD. Awards per officer vs race for the two cpd gender categories.} \label{fig:awards}
\end{figure}




% !TEX root = ../main.tex

\subsection{Building an officers' network}

Among several tasks, we can use the CPD dataset to perform network analysis. \textcolor{red}{Should cite policing papers who do network analysis?}. For example, we can use the \texttt{complaints\_officers.csv} data to construct an undirected graph $\mathcal{G} = \{\mathcal{V}, \mathcal{E}\}$, in which $\mathcal{V}$ is the set of nodes --- officers appearing in at least one complaint, and $\mathcal{E}$ the set of edges --- where an edge is present whenever two officers appeared on the same complaint. Moreover, we can link the complaints to the \texttt{tactical\_response\_reports.csv} file, to consider also the subgraph of officers who were also listed in a TRR. We report summary for the corresponding graphs in \Cref{tab:stats_graphs}. Here, we consider any complaint available to form an edge, but we only consider TRRs filed after January 1st 2004, and before December 1st 2015.

\begin{table}[h]
\begin{tabular}{c|c|c|c|c|c|c|c|}
\cline{2-8}
                                                & $|\mathcal{G}|$ & $|\mathcal{E}|$ & \textit{Avg. degree} & \textit{Triangles} & \textit{Max clique} & \textit{LCC} & \textit{\# Is. nodes} \\ \hline
\multicolumn{1}{|c|}{\textit{\textbf{All}}}     & $14{,}372$      & $106{,}701$     & $14.85$              & $361{,}878$        & $64$                & $13{,}950$   & $0$                   \\ \hline
\multicolumn{1}{|c|}{\textit{\textbf{In TRRs}}} & $4{,}105$       & $22{,}064$      & $10.75$              & $44{,}786$         & $28$                & $3{,}822$    & $225$                 \\ \hline
\end{tabular} \label{tab:stats_graphs}
\caption{Summary statistics for the complaints network graph, and the subgraph of officers in TRRs. Here LCC is the largest connected components, and Is. nodes is the number of isolated nodes.}
\end{table}

\begin{figure}[t!] 
	\includegraphics[width=\textwidth]{figs/degree_distribution} 
	\caption{Degree distribution for the complaints network.}
\label{fig:degree_distribution}
\end{figure}


\section{Discussion}

\subsection{Intended Use}
Benchmark algorithms for:
\begin{itemize}
	\item community detection (where for example unit assignment can be used to
		have a ground truth
	\item network inference from contagion data (complaint network can be used
		as ground truth, shootings as the contagion)
	\item time series prediction over networks
\end{itemize}

Relatedly, useful to assess and design contagion models


\textbf{TODO:} the following was copy-pasted from Daria's email.

In my view, in discussing intended uses for the data set, we should probably focus attention at the outset on the benchmarking capacities for our data:

\begin{itemize}
	\item Algorithms to detect network spatial and temporal clustering
	\item Algorithms to fit models of contagion on the network (in addition to shooting generally, researchers can examine other behaviors independently documented, like foot-chase shootings, use of force during arrest, abusive charging behavior, but also laudatory behavior like complying with reporting requirements, etc.)
\end{itemize}

Other uses for this dataset can include:

\begin{itemize}
	\item Correlating police behavior to police traits and complainant traits (like race, gender, location of assignment/residence, years on force, etc.)
	\item Correlating complaint filing frequency to police traits and complainant traits (see above)
	\item Creating network graphs for subunits of interest (particular police patrol districts, for example).
	\item Tracking complaint trends relative to external events like new disciplinary practices, new training techniques, new oversight (Department of Justice, police union monitoring, new civilian oversight, publication of shootings in major papers) and other events like high-profile scandals, introduction of new technologies like Tasers, etc.
	\item Subject to caveats, investigating how to control for the dangerousness of police work when investigating the frequency of particular police behaviors (use of excessive force, wrongful arrest, abusive speech, etc.)
\end{itemize}

\subsection{Limitations}

todo

%
%\section{Constructing the Network}\label{sec:data}

\textbf{TODO} clean, sections 4.1 is now very redundant with previous sections.
Important section here is 4.4 (Trevor stuff!)

\subsection{Data sources and cleaning}

{\color{red} this section 2.1 should include a description of the 6 datasets we used---2 complaints, 2 shootings, roster, and unit reference---and how they were constructed. I did my best to pull the important info from the chicago manuscript here but I probably did a bad job }

{\color{red} right now this is a hodge podge of info I could find about how we dealt with the original data.}

The original dataset used in this study was a collection of
107,276 {\color{red} civilian?} complaints against officers of the CPD
from the years {\color{red} xx to xx}, and cover a range of types of
allegations, including First Amendment violations, wrongful arrest, illegal
search and seizure, excessive force, and more.
The dataset---which is now publicly available at \url{https://invisible.institute/police-data}---was 
originally obtained  by the Citizens Police Data Project at the Invisible
Institute and the University of Chicago Law School’s Mandel Legal Aid Clinic
via a Freedom of Information Act request and subsequent litigation \cite{xx}.

To guard against endogeneity problems,
we excluded all complaints  {\color{red}(\_\%)} connected in any way to the discharge of a
firearm. {\color{red} did we?}

For the period under study, the Independent Police Review Authority (IPRA)
handled initial responsibility for processing and investigating civilian
complaints. Filing a complaint
required civilians to swear out an affidavit attesting to the truth of their
allegations. Filing also required affidavits to be signed in person, at a
limited number of locations. After a filing was complete, the complaint was
investigated by an independent authority or other agency. For those
investigations that were completed, allegations were either sustained,
un-sustained or the officer was exonerated. 
Less than 3\% of complaints were
sustained, and officers were almost never disciplined. 




We first cleaned the civilian complaints to fill in missing information from
{\color{red} other records??? how?? what records?}. A significant number of complaints were missing any officer
identities, race, gender, dates of appointments and other information, and
another group of complaints were missing affidavits from the complainants. Our
dataset included 19,843 complaints involving 9,737 unique officers that had
complete information. {\color{red} after cleaning? or did we just throw out data with missing entries}

We obtained personnel records on all Chicago police officers for the relevant
time period January of 2008 to November 2015 by filing a Freedom of Information
Act (FOIA) request with City of Chicago Department of Human Resources. The
bi-annual personnel data include each officer's name, position title (rank),
and original hire date. We use the officers first and last name and the
original date of appointment to link records with the other datasets. 
{\color{red} all these dates seem wrong too...}
 
 We linked information across datasets (the civilian complaint dataset and the
shootings dataset) by constructing a unique identifier for each and every
officer on the roster. In the event that officers shared names, we verified the
officer’s identity through badge numbers and/or date of appointment where
available on either the relevant complaints or the shootings data from IPRA.

Officers listed in the complaints were identified using personnel records on
all Chicago police officers that had been separately obtained by filing a
Freedom of Information Act (FOIA) request with the City of Chicago Department
of Human Resources. The bi-annual personnel data include each officer's name,
position title (rank), and original hire date. These records were available
from 2002 to 2014. We used the officers’ badge number and the original hire
date to link records with the complaint and shooting datasets, and to identify
unique officers. 

\subsection{Officer social network}


We constructed a social network of CPD police officers over which a possible
contagion effect could be mediated
using the {\color{red} two complaints datasets with 19,843 complaints between Jan 2008 and Nov 2015}. 
In particular, we constructed a single undirected network that 
 connects every pair of officers who were listed together on any
 civilian complaint during the period of study. This complaint network
contains 9,737 unique officer nodes and 47,037 edges. We did not
weight the edges of our complaint network to reflect the number of times that a
pair of officers are listed together on the same civilian complaint. The vast
majority (83\%) of edges would have a weight of 1, corresponding to officers who 
appeared together on a complaint only once. The average officer received 1.72 complaints. 
Approximately 15\% of officers connect to 10 or more edges in the network, while the top
1\% of officers connect to 26 edges on average. {\color{red} replace this with a degree distribution plot}

The network contains several smaller components, many isolated
officers, and one very large connected component. As is true for most social network analysis, we use this
largest connected component as the focus of our study, and discard the remainder. 
The largest connected component in the complaint network contains 7,825 (80\%) of the
nodes, 46,568 (99\%) of the edges, and 418 (93\%) of the 450 shooting officers in %of the 450
the original larger network. 
As detailed in the Appendix, the LCC also has a similar
clustering coefficient, an average path length, and its degree distribution
follows a power-law distribution as is true for the larger network. 
{\color{red} again need plots}

From this point onward, the ``complaint''/``social network''  refers to the largest connected component.

\paragraph{Rationale and limitations}
If two officers are listed on a complaint together, they respond
to calls together and---as evidence suggests---engage in misconduct
together. Being listed on a complaint together also suggests a pre-existing
relationship through which scripts of violence could be transmitted.
Previous research suggests that
co-offending represents strong and enduring relationships between individuals \cite{Xx}.
We therefore treated co-offending as evidence of an existing relationship
between the two individuals involved, rather than as a point-in-time estimate
or marker of when that relationship formed. Thus, the social network we build
is \emph{static} over time: an edge indicates that two individuals have co-offended together
at least once at some time during the study period.

We study civilian-facing complaints in particular based on the assumption that 
officers who interact during civilian encounters are
significantly more likely to spread civilian-related response behaviors---such as violence---through
those interactions. In our data, 60.3\% of complaints were filed by civilians,
while the remaining 39.7\% were generated from within the department. 
Over 55\% of civilian complaints in our dataset name two or more officers,
indicating that the majority of alleged police misconduct occurs in a group
setting.

These data were not without limitations.
Civilians are generally not reliable reporters of the events in question in a complaint allegation;
reports suffer from civilian biases against the police, faulty memory and other
common sources of mistake or misreporting. 
Further, a large portion of potential police misconduct goes unreported by
civilians, frequently owing to administrative requirements but also owing to a
range of factors associated with a strong distrust of authority, a lack of
faith in accountability, and a reluctance to risk retaliation.  
We make no claims about civilian
complaints other than that the officers that are listed as subjects of the
complaints interacted with each other and with civilians in the encounter at
the reported place on the reported date. 
But even so, the complaint
co-offender network cannot be said to accurately represent the social
connections among officers; we capture a subset---but not the full set---of 
social connections among officers.

Finally, the subset of officers in the complaint network are not necessarily
representative of the larger population of police officers. If officers in the
complaint network differ from those not in the network, our results might not
represent actual contagion for the department. However, representativeness does 
not pose a strong concern because the majority of officers (70-75\%) and shooting officers (93\%)  are present in our 
network {\color{red} true of the LCC?}. Thus, the analysis captures a significant
fraction of both shooting and non-shooting officers. 
 But we cannot determine whether this network
is representative of other police departments outside of Chicago. 

\subsection{Officer shooting history}

We endowed each officer node in the social network with a history of shooting behaviour
using detailed incident-level data {\color{red} two shooting datasets}
collected by IPRA on police-involved shootings occurring between 2008 and
November 2015. Chicago city regulations required IPRA to investigate ``all cases
in which a department member discharges his or her firearm, stun gun, or Taser
in a manner which potentially could strike an individual, even if no allegation
of misconduct is made.'' CPD General Orders required officers to notify IPRA
each time a CPD member discharged a firearm. Once notified, IPRA created an
electronic record of its investigation by assigning each incident a log number
in CPD and IPRA’s electronic case management system (CLEAR). 

City regulations required IPRA to issue quarterly reports on weapons-discharge
investigations. In its reports, IPRA included five categories of
weapons-discharge notifications: “hit shooting” of firearm, “non-hit shooting”
of firearm, shooting of firearm at an animal, shootings with taser, and
oleoresin capsicum (OC). We concentrate analysis on the first category of data
consisting of police-involved “hit shootings” (meaning shootings that hit a
civilian) with a firearm, from January 2008 to November 2015. 

Reports provided detailed incident-level information from these categories. Key
variables in this dataset included: the date and location of the shooting,
identifying information associated with the officers who had discharged their
gun, and information about the organizational assignment of the relevant
officer. Individual officers involved in police shootings were linked through a
unique identifying number to those officers in civilian complaints.

We excluded both same-day same-event shootings, involving connected officers
who were involved in a police shooting at the same event, and same-day
different event shootings. Excluding the latter (for now) biases our
investigation in a conservative direction against a finding; same-day different
event shootings are likely to reflect contagion, particularly in light of the
short period of time between shootings.

In the shootings network, nodes in the graph represent individual shootings,
meaning a discharge of a firearm by an individual officer. Each node in the
graph pairs an individual shooting with the date of the shooting and the
identity of the shooting officer (n= 488 shootings/officers). Because a number
of officers committed more than one shooting, the shootings network contained a
smaller number of unique officers (o = 418 unique officers).

Table 1 summarizes the complaint and shootings data:

\begin{figure}
\includegraphics[width=0.4\textwidth]{figs/complaints_shootings_table_0.png}
\caption{table 1}
\end{figure}


In addition, the IPRA dataset reporting police-involved shootings has several
important limitations. Most notably, subsequent analysis by the City of
Chicago’s Office of Inspector General (OIG) found that IPRA did not follow best
practices in reporting use of force. Among other difficulties, IPRA relied on
CPD notifications, and did not independently verify that the Department had
provided all of the required weapons-discharge notifications.  At the same
time, the data appears relatively reliable. The OIG report compared the IPRA
reported shootings to those reported in CPD internal “tactical response
reports.” The comparison found that IPRA’s shooting data from September 2007 to
September 2014, which reported 488 “hit shootings,” had overreported by only
four shootings compared to the CPD tactical response reports, an error rate we
deemed to be within an acceptable range. 

We used IPRA incident-level shooting data to layer shooting events occurring
during the period under study onto the complaint network of co-listed officers
described above. For each shooting event, we recorded the date and location
(street address) of the shooting. We added this information to the complaint
network by adding the shooting event data to the officer-nodes in the network,
and in particular, to the officers who were recorded as having committed the
shootings. 

Fig. 2  Construction of the shooting network. On the left, the complaint network with shooting officers in red; on the right, the resulting shooting network.

In contrast to the complaint network, the shootings network is an event-focused
network. Some officers are involved with more than one shooting. A significant
fraction (86\%, n=358) of officers in the complaint network were associated
with only one shooting. A smaller fraction (12\%, n=51) of officer nodes were
associated with two shootings; 2\% (n=8) were involved in three shootings, and
one officer (Tracey Williams) was involved in five shootings. All of these
shootings are represented in the shootings network. 


\begin{figure}
\includegraphics[width=0.4\textwidth]{figs/complaints_shootings_table.png}
\caption{table 2}
\end{figure}



%log_number,officer_name,unique_identifier,incident_date,incident_address,incident_year,specific_day_time


%Complaint_Number,Beat,Location_Code,Address,Street,Apartment,City_State_Zipcode,Incident_Datetime,Complaint_Date,Closed_Date,Full_Address,Investigator_Name,Investigator_Current_Assignment,Investigator_Rank,Investigator_Star,Investigator_Date_Appointed,Accused_Name,Accused_Birth_Yr,Accused_Gender,Accused_Race_Code,Accused_Date_Appointed,Accused_Current_Unit,Accused_Current_Rank,Accused_Star,Accused_Complaint_Category,Accused_Finding,Accused_Recommended_Discipline,Accused_Final_Finding,Accused_Discipline,PO_Witness_Name,PO_Witness_Gender,PO_Witness_Race,PO_Witness_Star,PO_Witness_Birth_Year,PO_Witness_Date_Appointed,Victim_Gender,Victim_Age,Victim_Race_Desc,Complainant_Gender,Complainant_Age,Complainant_Race_Desc

\subsection{Officer covariates}


% the roster data (as of April 2017):
%row_id,gender,race,birth_year,current_age,current_status,appointed_date,rank_no,current_rank,current_unit,unit_description,resignation_date,star1,star2,star3,star4,star5,star6,star7,star8,star9,star10,first_name,first_name_NS,last_name,last_name_NS,middle_initial,middle_initial2,suffix_name,merge,roster_1936-2017_2017-04_ID,UID,old_UID,link_UID

\begin{figure}
\centering
\includegraphics[width=\textwidth]{figs/data_inspection.pdf}
\caption{histograms of officer covariate}\label{fig:histograms}
\end{figure}

\begin{figure}
\centering
\includegraphics[width=\textwidth]{figs/shooters_relative.pdf}
\caption{histograms of shooter covariate (relative)}\label{fig:shooterhist}
\end{figure}

For each officer, the following information was available in our data: full
name, gender, race, birth date, appointment date, resignation date, rank, badge
number(s), and full unit membership history (including unit number, beginning date, and end date for each membership record).  
We used five of these variables---gender, race, birth date, rank, and number of years on the force since appointment---directly
as covariates in our statistical analyses. 
Figure \ref{fig:histograms} shows the distribution of values for each of these three covariates in the largest
connected component of the officer network. 
As we do not expect an officer's name or badge number to have any additional effect on their likelihood of shooting,
we did not use these variables as covariates. 
The sixth and final covariate for each officer that we considered is their unit membership history.
This section details how we processed each officer's unit membership
history to form a covariate for each officer, and then how we combined this with the previous covariates.

Our data contained a list of 242 unique units with a name and number. After removing
units that were not joined by any officer in the largest connected network component---many of them
old units that no longer exist---179 remained. We separated the units into groups
based on their function, to attempt to ensure a consistent effect on the likelihood of officer shootings in each group. 
The original unit reference data did not contain any actual description of unit function; for those
whose purpose was not clear from the name or whose name was missing entirely, 
we used the following sources to investigate unit function by number:
\begin{itemize}
\item Police directives from \url{http://directives.chicagopolice.org} and \url{https://directives.crimeisdown.com}
\item Investigatory Stop Report (ISR) counts by unit from the CPD annual reports 2017, 2018
\item Tactical Response Report (TRR) counts by unit from the CPD annual reports 2017, 2018
\item Chicago Data Portal crime counts: \url{https://data.cityofchicago.org/Public-Safety/Crimes-2001-to-Present/ijzp-q8t2}
\item Unit mentions in annual reports: \url{https://home.chicagopolice.org/statistics-data/statistical-reports/annual-reports/}
\item Specialized unit information: \url{https://home.chicagopolice.org/about/specialized-units/}
\item Organization charts: \url{https://www.chicago.gov/content/dam/city/depts/cpb/SuperintendentSearch/CPDOrgChart.pdf},
\url{https://home.chicagopolice.org/wp-content/uploads/2020/01/Department-Organization-for-Command-2019-July-19.pdf }
\end{itemize}

The resulting grouping of units is as follows. In total, we grouped the units into 
14 groups, plus an additional extra group for unit numbers appearing in the network whose
function and/or name are unknown. Some units appeared with duplicate numbers to others
(labelled ``dup.'' in the unit tables), but the duplicate name had the same meaning as the original
and did not influence grouping. 
\begin{description}
\item[Low Intensity / Unarmed Officers (49 units, 1 group, Table \ref{tab:desk})] Units
engaging in non-public-facing and
non-criminal-facing activities, units with unarmed officers,
and units performing low-intensity activities {\color{red}as judged by ISR / TRR}.
%
\item[District Units (27 units, 4 groups, Table \ref{tab:district})] Units that
correspond to patrols in the 25 districts\footnote{{\color{red} as of recent;
these change over time. How did we handle this?}} of Chicago. Of these, 25 are district-specific, and
the remaining two are city-wide. Given that Chicago is strongly racially and
economically segregated by district, we decided not to treat all district
patrols as a single unit type in the data, but rather clustered the district
units into 4 groups based on the number of homicides within each district
during the study period using the $k$-means algorithm \cite{xx}.  Figure
\ref{fig:districtclusters} shows the clustering of districts by number of
homicides, and Figure \ref{fig:districtelbow} shows the ``elbow plot'' used
to decide on the number of clusters. 
The two remaining units in this group are the bureau of patrol and
recruit training units; we treated officers in these units as city-wide
patrols, and used the average number of homicides city-wide to cluster these
units.
%
\item[Area Units (33 units, 3 groups, Table \ref{tab:area})] Units that operate within
the 6 areas\footnote{{\color{red} as of recent; these change over time. How did we handle this?}}  of
Chicago.  These include area patrols, youth divisions, gang and narcotics enforcement,
and violent and property crimes. We subdivided the area-based units similarly
to the district-based units due to the strong racial and economic segregation
of the city; in particular, we clustered the area units into 3 groups based on
the number of homicides in the area during the study period using the $k$-means
algorithm \cite{xx}. Figure \ref{fig:areaclusters} shows the clustering of
areas by number of homicides, and Figure \ref{fig:areaelbow} shows the ``elbow plot'' used
to decide on the number of clusters.
%
\item[Detail \& Controlled Zone Security (16 units, 2 groups, Table \ref{tab:controlzone}) ] Units whose officers
provide security of ``controlled zones'' (e.g., airports, courts, construction sites). Of these,
5/11 were labelled ``high/low-intensity'' {\color{red} based on ISR/TRR}.
%
\item[City-Wide Investigations (29 units, 2 groups, Table \ref{tab:citywide})] Units whose officers
perform city-wide detective, investigatory, and enforcement work. 
Of these, 16/13 were labelled ``high/low-intensity'' {\color{red} based on ISR/TRR}.
One unit worth mentioning is the Juvenile Intervention Support Center (JISC); although
we placed this unit in the ``low-intensity'' group, it 
could have equally been sorted into the ``high-intensity'' category, based
on its middling ISR and TRR counts.
%
\item[Special Operations (12 units, 2 groups, Table \ref{tab:specops})] Units whose officers
engage in special operations (e.g., SWAT, mounted, mobile strike force).
Of these, 9/3 were labelled ``high/low-intensity'' {\color{red} based on ISR/TRR}.
%
\item[Unknown (13 units, 1 group, Table \ref{tab:unk})] Units that are labelled as unknown in our dataset, and/or for which
we could not find any credible source of information detailing either the name or purpose of the unit. Given
naming/numbering patterns that are present in the remainder of the data, we hypothesize that units 271--5 are 
area-specific special operations units; but we could not verify this with any existing records. Although unit 45 had
a known label (``district reinstatement''), we could not find any source that described its purpose.

\end{description}



Next, we calculated the fraction of each officer's career spent being a member of
each of these groups. Note that officers can be a member of one or multiple units at the same time, so
these fractions do not necessarily sum to 1. The final ``unit membership'' covariate for each officer
was the vector of these 14 values (all between 0 and 1), one for each unit group.

Finally, we discretized the 6 covariates (gender, race, birth date, rank, number of years on the force since appointment,
and 14 fractions of time spent in each unit group) into bins:
\begin{description}
\item[Gender] 2 values were present in the data: male and female.
\item[Race] 6 values were present in the data: hispanic, white, black, asian/pacific islander, indian, and other. {\color{red} these correct?} 
\item[Birth Date] We grouped birth years together by 5 year intervals, resulting in {\color{red} XX bins}.
\item[Rank] {\color{red}XX values were present in the data: ...}
\item[Years on the Force] We grouped years on the force into 8 bins: 1, 2, 3-5, 6-10, 11-20, 21-30, 31-40, and 40+.
\item[Unit Membership] We discretized all 14 fractions into spacings of 0, 0.1, 0.2, etc. This created $10^{14}$ bins.
\end{description}
Combining all of these covariates, each officer was described by one of $xx$ unique values. Only officers with precisely
the same value were eligible to be swapped in the permutation test. 

Figure \ref{xx} shows the occupancy of bins

{\color{red} there were also units that appeared in our dataset but no officer was a recorded member of them in the network? LCC? here's the list of those}



\begin{figure}[h!]
\centering
\includegraphics[width=0.75\textwidth]{figs/elbow-districts.pdf}
\caption{}\label{fig:districtelbow}
\end{figure}

\begin{figure}[h!]
\centering
\includegraphics[width=0.75\textwidth]{figs/clusters-districts.pdf}
\caption{}\label{fig:districtclusters}
\end{figure}

\begin{figure}[h!]
\centering
\includegraphics[width=0.75\textwidth]{figs/elbow-areas.pdf}
\caption{}\label{fig:areaelbow}
\end{figure}

\begin{figure}[h!]
\centering
\includegraphics[width=0.75\textwidth]{figs/clusters-areas.pdf}
\caption{}\label{fig:areaclusters}
\end{figure}




\begin{table}
\tiny
\centering
\caption{}\label{tab:desk}
\begin{tabular}{|ll|}
\hline
Type: &	\textbf{Very Low Intensity \& Unarmed Officers} \\
\hline
Unit \# &	Unit Name \\
\hline
26	&Executive Officers Unit\\
86	&OEC-Police Dispatch\\
102	&Office Of News Affairs\\
111	&Office Of The Superintendent\\
112	&Bureau Of Professional Standards\\
114	&Legal Affairs Section\\
115	&CAPS Project Office\\
115	&Crime Control Strategies Section\\
116	&Deployment Operations Center\\
119	&Office Of International Relations\\
120	&Bureau Of Support Services\\
121	&Bureau Of Internal Affairs\\
123	&Human Resources Division\\
125	&Information Services Division\\
127	&Research And Development Division\\
128	&Professional Counseling Division\\
129	&Management And Labor Affairs Section\\
130	&Bureau Of Organizational Development\\
130	&Technology And Records Group\\
130	&Bureau Of Staff Services\\
163	&Records Inquiry Section\\
167	&Evidence And Recovered Property Section\\
169	&Police Documents Section\\
170	&CT \& ID Administration\\
172	&Equipment And Supply Section\\
175	&Telecommunications Unit\\
176	&Communication Operations Unit\\
214	&Freedom Of Information Section\\
216	&Deputy Chief Central Control Group\\
179	&Reproduction And Graphic Arts Section\\
231	&Medical Section\\
159	&Gun Registration\\
157	&Pub Housing Div Adm\\
601	&Det Div Admin.\\
944	&Cops Grant Recr Trng\\
165	&Field Inquiry Section\\
161	&General Support Division\\
139	&Asst Superintendent-Law Enforcement Operations (dup. CAPS)\\
140	&Office Of The First Deputy Superintendent\\
376	&Alternate Response Section\\
136	&Special Events Unit\\
166	&Field Services Section\\
126	&Inspection Division\\
147	&Senior Citizen Services\\
154	&Traffic Safety And Training\\
168	&Auto Pound Section\\
441	&Special Activities Section\\
135	&Chicago Alternative Policing Strategy (CAPS) Division\\
136	&CAPS\\
\hline
\end{tabular}
\end{table}



\begin{table}
\tiny
\centering
\caption{}\label{tab:district}
\begin{tabular}{|llll|}
\hline
Type: &	\textbf{District-Specific Units}  & &\\
\hline
Unit \# &Unit Name & Homicides & Cluster  \\
\hline
1	&District 001	&100	&0\\
2	&District 002	&455	&1\\
3	&District 003	&662	&2\\
4	&District 004	&693	&2\\
5	&District 005	&658	&2\\
6	&District 006	&780	&2\\
7	&District 007	&931	&3\\
8	&District 008	&644	&2\\
9	&District 009	&666	&2\\
10	&District 010	&700	&2\\
11	&District 011	&1055	&3\\
12	&District 012	&385	&1\\
13	&District 013	&385	&1\\
14	&District 014	&249	&0\\
15	&District 015	&626	&2\\
16	&District 016	&85	&0\\
17	&District 017	&149	&0\\
18	&District 018	&113	&0\\
19	&District 019	&144	&0\\
20	&District 020	&69	&0\\
21	&District 021	&455	&1\\
22	&District 022	&352	&1\\
23	&District 023	&144	&0\\
24	&District 024	&167	&0\\
25	&District 025	&521	&1\\
142	&Bureau Of Patrol	&447.52	&1\\
44	&Recruit Training Section	&447.52	&1\\
\hline
\end{tabular}
\end{table}



\begin{table}
\tiny
\centering
\caption{}\label{tab:area}
\begin{tabular}{|llll|}
\hline
Type: &	\textbf{Area-Specific Units}  & &\\
\hline
Unit \# &Unit Name & Homicides & Cluster  \\
\hline
71	&Youth Division Area1	&1672	&0\\
72	&Youth Division Area2	&2483	&1\\
73	&Youth Division Area3	&2241	&1\\
74	&Youth Division Area4	&2525	&1\\
75	&Youth Division Area5	&1630	&0\\
76	&Youth Division Area6	&637	&0\\
211	&Bureau Of Patrol - Area Central&3725	&2\\
212	&Bureau Of Patrol - Area South	&3414	&2\\
213	&Bureau Of Patrol - Area North	&3065	&2\\
214	&Deputy Chief - Area 4 (dup. Free. Of Info. Divn)&2625	&1\\
62	&Area 2 Pat Narc Prog	&2483	&1\\
63	&Area 3 Pat Narc Prog	&2241	&1\\
65	&Area 5 Pat Narc Prog	&1630	&0\\
66	&Area 6 Pat Narc Prog	&637	&0\\
311	&Gang Enforcement - Area Central&3151	&2\\
312	&Gang Enforcement - Area South	&3145	&2\\
313	&Gang Enforcement - Area North	&637	&0\\
314	&Gang Section - Area 4	&2625	&1\\
315	&Gang Section - Area 5	&1630	&0\\
612	&Violent Crimes DDA 1	&3151	&2\\
610	&Detective Area - Central	&3725	&2\\
620	&Detective Area - South	&3414	&2\\
621	&Prop Crimes DDA 2	&3145	&2\\
622	&Violent Crimes DDA 2	&3145	&2\\
630	&Detective Area - North	&3065	&2\\
631	&Prop Crimes DDA 3	&637	&0\\
632	&Violent Crimes DDA 3	&637	&0\\
640	&Detective Section - Area 4	&2625	&1\\
641	&Prop Crimes DDA 4	&2625	&1\\
642	&Violent Crimes DDA 4	&2625	&1\\
650	&Detective Section - Area 5	&1630	&0\\
651	&Prop Crimes DDA 5	&1630	&0\\
652	&Violent Crime DDA 5	&1630	&0\\
\hline
\end{tabular}
\end{table}



\begin{table}
\tiny
\centering
\caption{}\label{tab:controlzone}
\begin{tabular}{|ll|}
\hline
Type:	&\textbf{Detail \& Controlled Zone Security - High Intensity}\\ 
\hline
Unit \#	&Unit Name \\
\hline
704	&Transit Security Unit\\	
701	&Public Transportation Section \\
50	&Airport Law Enforcement Section - North \\ 
51	&Airport Law Enforcement Section - South \\
171	&Central Detention Unit	\\ 
\hline
Type:	&\textbf{Detail \& Controlled Zone Security - Low Intensity}\\ 
\hline
Unit \#	&Unit Name \\
\hline
143	&Court Section (dup .District Law)\\
261	&Court Section\\
284	&Admin School Securit\\
541	&FOP Detail\\
542	&Detached Services - Goverment Security\\
543	&Detached Services - Miscellaneous Detail\\
276	&OEC - Detail Section\\
57	&Traffic Section Detail Unit (dup. Detail Unit)\\
151	&Traffic Enforcement\\
152	&Loop Traffic Unit\\
145	&Traffic Section\\
\hline
\end{tabular}
\end{table}


\begin{table}
\tiny
\centering
\caption{}\label{tab:citywide}
\begin{tabular}{|ll|}
\hline
Type:	&\textbf{City-wide Investigations - High Intensity}\\ 
\hline
Unit \#	&Unit Name \\
\hline
189	&Narcotics Division\\
91	&Narc Special Enforce\\
92	&Narc General Enforce\\
156	&Gang Crime Section\\
393	&Gang Enforcement Division\\
188	&Bureau Of Organized Crime\\
193	&Gang Investigation Division\\
192	&Vice \& Asset Forfeiture Division\\
196	&Asset Forfeiture Investigation Section \\
740	&G/C Unit West\\
760	&G/C Unit North\\
710	&G/C Unit South\\
765	&Public Housing North\\
715	&Public Housing South\\
132	&Preven \& Neigh Div\\
606	&Central Investigations Division\\
\hline
Type:	&\textbf{City-wide Investigations - Low Intensity}\\
\hline
Unit \#	&Unit Name \\
\hline
79	&Special Investigations Unit\\
180	&Bureau Of  Detectives\\
184	&Youth Investigation Division\\
608	&Major Accident Investigation Unit\\
602	&Central Auto Theft\\
603	&Arson Section\\
603	&Bomb And Arson Division\\
384	&Juvenile Intervention Support Center (JISC)\\
191	&Intelligence Section\\
277	&Forensic Services Evidence Technician Sectn\\
377	&Forensic Services Unit - ET North\\
477	&Forensic Services Unit - ET South\\
177	&Forensic Services Division\\
\hline
\end{tabular}
\end{table}


\begin{table}
\tiny
\centering
\caption{}\label{tab:specops}
\begin{tabular}{|ll|}
\hline
Type:	&\textbf{Special Operations - High Intensity}\\ 
\hline
Unit \#	&Unit Name \\
\hline
55	&Mounted Unit\\
58	&Spec Func Canine\\
341	&Canine Unit\\
353	&Special Weapons And Tactics (SWAT) Unit\\
153	&Mobile Strike Force (dup. Special Functions Support Unit)\\
141	&Special Functions Division\\
146	&Canine Unit\\
241	&Troubled Building Unit\\
253	&Targeted Response Unit\\
\hline
Type:	&\textbf{Special Operations - Low Intensity}\\
\hline
Unit \#	&Unit Name \\
\hline
59	&Marine Operations Unit\\
442	&Bomb Squad\\
124	&Education And Training Division\\
\hline
\end{tabular}
\end{table}


\begin{table}
\tiny
\centering
\caption{}\label{tab:unk}
\begin{tabular}{|ll|}
\hline
Type: &	\textbf{Unknown} \\
\hline
Unit \# &	Unit Name \\
\hline
45  & District Reinstatement \\
52  & Unknown\\
54  & Unknown\\
56  & Unknown\\
64  & Unknown\\
661 & Unknown\\
720 & Unknown\\
271 & Unknown (Special Funcs Area 1?)\\
272 & Unknown (Special Funcs Area 2?)\\
273 & Unknown (Special Funcs Area 3?)\\
274 & Unknown (Special Funcs Area 4?)\\
275 & Unknown (Special Funcs Area 5?)\\
911 & Unknown \\
\hline
\end{tabular}
\end{table}




{\color{red}
\paragraph{Questions:}
\begin{itemize}
\item for years on the force -- this is appointment date minus what?  \tbo{end of study period minus appointment date}
\item what did we do with officers in the LCC who did not match?
\item did we include the UNK bins? how? -- yes, the unk fraction is treated as another "component" in the vector covariate
\item how did we deal with changing areas over time? changing districts?
\item 4 clusters in area unit plot, but only 3 unit bins
\end{itemize}
}

\begin{figure}[h!]
\centering
\includegraphics[width=0.75\textwidth]{figs/unit_plots.pdf}
\caption{}\label{fig:unitplots}
\end{figure}

Figure \ref{fig:unitplots} shows the number of shooting events
performed by officers in each unit and each unit bin during the study
period. To construct this figure, note that there are three
situations for each shooting event:
\begin{itemize}
\item The shooting officer is not present in the largest connected component of our network. 
These events are not included in the study, but we show them here labelled
``N/C'' (not counted). There are 38 such shooting events by 35/484 unique officers.
\item The shooting officer is present in the largest connected component but we could
not find a positive match for the officer in the roster (which has the unit history). 
These events are labelled ``N/A'' (not available). There are 69 such shooting events
by 57/484 unique officers.
\item The shooting officer is present in the largest connected component and there is
a positive match in the roster. There are 452 such shooting events by 392/484 unique officers.
Note that it is possible that a shooting officer is a member of multiple units simultaneously;
in this case, we add +1 to the shooting event counts for every unit the officer was a member of
at the time of shooting. But there is only one shooting event with such a situation, % Thomas Dineen 
where the officer was a member of both units 15 and 153.
\end{itemize}



\bibliographystyle{unsrtnat}
\bibliography{references}

\input{sections/checklist}

\appendix



\section{Datasheet}

\subsection{Motivation}

\paragraph{For what purpose was the dataset created?}

\paragraph{Who created the dataset, and on behalf of which entity?}

\paragraph{Who funded the creation of the dataset?}

\paragraph{Any other comments?}

\subsection{Composition}

\paragraph{What do the instances that comprise the dataset represent?}

\paragraph{How many instances are there in total (of each type)?}

\paragraph{Does the dataset contain all possible instances or is it a sample of instances from a larger set?}

\paragraph{What data does each instance consist of?}

\paragraph{Is there a label or target associated with each instance?}

\paragraph{Is any information missing from individual instances?}

\paragraph{Are relationships between individual instances made explicit?}

\paragraph{Are there recommended data splits?}

\paragraph{Are there any errors, sources of noise, or redundancies in the dataset?}

\paragraph{Is the dataset self-contained, or does it rely on external resources?}

\paragraph{Does the dataset contain data that might be considered confidential?}

\paragraph{Does the dataset contain data that, if viewed directly, might be offensive, insulting, or threatening?}

\paragraph{Does the dataset relate to people?} 

\paragraph{Does the dataset identify any subpopulations?}

\paragraph{Is it possible to identify individuals?}

\paragraph{Does the dataset contain data that might be considered sensitive in any way?}

\subsection{Collection Process}

\paragraph{How was the data associated with each instance acquired?}

\paragraph{What mechanisms or procedures were used to collect the data?}

\paragraph{If the data are a sample from a larger set, what was the sampling strategy?}

\paragraph{Who was involved in the data collection process and how were they compensated?}

\paragraph{Over what timeframe was the data collected?}

\paragraph{Were any ethical review processes conducted?}

\paragraph{Does the dataset relate to people?}

\paragraph{Did you collect the data from the individuals directly, or obtain it via third parties?}

\paragraph{Were the individuals notified about the data collection?}

\paragraph{Did the individuals in question consent to the collection and use of their data?}

\paragraph{If consent was obtained, were the consenting individuals provided with a mechanism to revoke their consent in the future or for certain uses?}

\paragraph{Has analysis of the potential impact of the dataset and its use on data subjects been conducted?}

\subsection{Preprocessing and cleaning}

\paragraph{Was any preprocessing of the data done?}

\paragraph{Was the ``raw'' data saved in addition to the cleaned data?}

\paragraph{Is the software used to clean the data available?}

\subsection{Uses}

\paragraph{Has the dataset been used for any tasks already?}

\paragraph{Is there a repository that links to any or all papers that use the dataset?}

\paragraph{What (other) tasks could the dataset be used for?}

\paragraph{Is there anything about the composition of the dataset or the way it was collected and cleaned that might impact future uses?}

\paragraph{Are there tasks for which the dataset should not be used?}

\subsection{Distribution}

\paragraph{Will the dataset be distributed to third parties outside of the entity on behalf of which the dataset was created?}

Yes, the data is publicly available.

\paragraph{How will the dataset be distributed?}

It is available on GitHub at \url{https://github.com/chicago-police-violence/data}.

\paragraph{When will the dataset be distributed?}

It is currently publicly accessible.

\paragraph{Will the dataset be distributed under a copyright, other IP license, or terms of use?}

Yes; the source code is released under the MIT license, and the data output by the cleaning code is released under the Creative Commons 4.0 BY-NC-SA license.

\paragraph{Have any third parties imposed IP-based or other restrictions on the data associated with the instances?}

No.

\paragraph{Do any export controls or other regulatory restrictions apply to the data?}

No.

\subsection{Maintenance}

\paragraph{Who is supporting/hosting/maintaining the dataset?}

The repository will be hosted on GitHub. As of August 2021, the repository
owners are Thibaut Horel, Trevor Campbell, and Lorenzo Masoero, but ownership
may change over time.

\paragraph{How can the data owner/curator be contacted?}

Issue threads on GitHub are the primary channel of contact for the repository maintainers.

\paragraph{Is there an erratum?}

Not as of yet. For each major release version, notes will be included and
hosted in the repository that will detail cleaning/linking errors that have been fixed.

\paragraph{Will the dataset be updated?}

The original raw source data from FOIA requests will not be modified. More raw
data files may be added over time corresponding to new FOIA requests. The data
cleaning and linking code will be edited over time to fix errors; release
versions will be clearly marked on GitHub.

\paragraph{If the dataset relates to people, are there applicable limits on the retention of data associated with the instances?}

No; this data was released per FOIA requests and is in the public domain.

\paragraph{Will older versions of the dataset continue to be supported/hosted/maintained?}

Yes; a full version-controlled history of the project will exist on GitHub.

\paragraph{If others want to extend/augment/build on/contribute to the dataset, is there a mechanism for them to do so?}

Yes; the repository for the dataset is hosted on GitHub, where pull requests are a usual channel for external contribution.


\section{Additional summary figures}\label{sec:additional_figs}

In this section we provide additional exploratory visualizations 
of the data. \cref{fig:trrs_times} provides temporal information
about TRRs, \cref{fig:position} shows the fraction of officers 
in different positions by demographic, \cref{fig:salary_gender_race}
shows officer salary by demographic, 
and \cref{fig:complaints_times} provides additional temporal information about complaints.
\cref{fig:degree_distribution,fig:intrashooting_time} display properties 
of the social network described in \cref{sec:discussion}.

\begin{figure}[ht!] 
	\includegraphics[width=\textwidth]{figs/trrs_times} 
	\caption{TRRs filed versus time at various scales (hour, weekday, month, year).} \label{fig:trrs_times}
\end{figure}
\hfill
\begin{figure}[ht!]
	\includegraphics[width=\textwidth, clip, trim= 0 0 460 0]{figs/complaints_times} 
\caption{Complaints versus time at various scales (weekday, month)}\label{fig:complaints_times}
\end{figure}

\begin{figure}[ht!] 
	\begin{center}
\begin{subfigure}{0.47\textwidth}
\includegraphics[width=\textwidth]{figs/position_race} 
\end{subfigure}
\hfill
\begin{subfigure}{0.47\textwidth}
\includegraphics[width=\textwidth]{figs/position_gender} 
\end{subfigure}
	\end{center}
	\caption{Fraction of officers in pools of ranks (Officer, Detective, Sergeant, Cap\-tain/Com\-mander/Lieutenant, and (First/Deputy) Chief/Superintendent)
 by race (left) and gender (right).} \label{fig:position}
\end{figure}

\begin{figure}[ht!] 
	\begin{center}
	\includegraphics[width=0.8\textwidth]{figs/salary_by_race_gender} 
	\end{center}
	\caption{Salary (box indicates quartiles, whiskers indicate 5th/95th percentile, and scatter points are outliers) by race and gender.} \label{fig:salary_gender_race}
\end{figure}


\begin{figure}[ht!] 
	\includegraphics[width=\textwidth]{figs/degree_distribution} 
	\caption{Degree distribution for the complaints network.}
\label{fig:degree_distribution}
\end{figure}

\begin{figure}[ht!] 
	\includegraphics[width=\textwidth]{figs/intrashooting_times} 
	\caption{Intrashooting times between pairs of shootings.}
\label{fig:intrashooting_time}
\end{figure}






\end{document}
