\documentclass{article}
\usepackage[nonatbib]{neurips_data_2021}
\usepackage[sort&compress,numbers]{natbib}
\usepackage[utf8]{inputenc}
\usepackage[urlcolor=blue,citecolor=blue,breaklinks,colorlinks]{hyperref}
\usepackage{url}
\usepackage[capitalize]{cleveref}
\usepackage{graphicx}
\usepackage{subcaption}
\usepackage{microtype,xcolor,booktabs}
\newcommand{\tbo}[1]{\textcolor{teal}{/* TBo: #1 */}}
\title{The CPD Data Set: Personnel, Use of Force, and Complaints in the Chicago Police Department}

\author{%
  Thibaut Horel\\
  LIDS, MIT \\
  \texttt{thibauth@mit.edu } 
  \And
  Lorenzo Masoero\\
  CSAIL, MIT \\
  \texttt{lom@mit.edu } 
   \And
 Raj Agrawal\\
  ArbiLex\\
  \texttt{r.agrawal@csail.mit.edu } 
   \And
  Daria Roithmayr\\
  Gould School of Law \\
  University of Southern California  \\
  \texttt{droithmayr@law.usc.edu}
 \And 
 Trevor Campbell \\
  Department of Statistics \\
   University of British Columbia\\
  \texttt{trevor@stat.ubc.ca}
}

\begin{document}


\maketitle

\begin{abstract}
\looseness=-1\relax
The lack of accessibility to data on policing has severely limited researchers'
ability to conduct thorough quantitative analyses on police activity and
behavior, particularly with regard to predicting and explaining police
violence. In the present work, we provide a new dataset that contains
information on the personnel, activities, use of force, and complaints in the
Chicago Police Department (CPD). The raw data, obtained from the CPD via a
series of requests under the Freedom of Information Act (FOIA), consists of 35
unlinked, inconsistent, and undocumented spreadsheets. Our paper provides a
cleaned, linked, and documented version of this data that can be reproducibly
generated via open source code. We provide a detailed description of the
dataset contents, the procedures for cleaning the data, and summary statistics.
The data have a rich variety of uses, such as prediction (e.g., predicting
misconduct from officer traits, experience, and assigned units), network
analysis (e.g., detecting communities within the social network of officers
co-listed on complaints), spatiotemporal data analysis (e.g., investigating
patterns of officer shooting events), causal inference (e.g., tracking the
effects of new disciplinary practices, new training techniques, and new
oversight on complaints and use of force), and much more. Access to this
dataset will enable the machine learning community to meaningfully engage with
the problem of police violence.
\end{abstract}

\textbf{Repository: \url{https://github.com/chicago-police-violence/data}}\\
\textbf{Instructions:} run \texttt{make} in the repository's root directory.
See \texttt{README.md} for detailed instructions.\\
\textbf{Data Set:} may be found in the \texttt{final/} directory after running \texttt{make} and is downloadable as a ZIP file at \url{https://github.com/chicago-police-violence/data/releases/download/v0.1-pre/cpd-dataset-v0.1-pre.zip}.

% !TEX root = ../main.tex


\section{Introduction} \label{sec:intro}

%%The lack of accessibility to data on policing has severely limited researchers’ ability to conduct thorough quantitative analyses on police activity and behavior, particularly with regard to predicting and explaining police violence. 


Over the last few decades, the problem of police brutality has emerged as a mainstream social and political issue. In the United States,  policymakers have identified the lack of data collection and scarcity of reliable downstream data analysis of policing activities as key elements of this problem. Indeed, improving data collection, and related analysis, comes with the promise of enhancing police, as well as promoting effective police reforms. For example, the 21st Century Task Force, a presidential initiative proposed in 2014 to build trust between citizens and their law enforcement officers, identified the lack of policing data as a major contributor to the inability of communities and law enforcement agencies to ``make informed policy and practice adjustments based on good information''. Relatedly, the task force highlighted the need to collect more and better data, to ``improve the level of trust, transparency, and accountability between communities and law enforcement agencies.''\footnote{\url{https://cops.usdoj.gov/RIC/Publications/cops-p341-pub.pdf}}. 

Small steps in this directions are underway: the Department of Justice (DOJ) is currently collecting data on arrest-related deaths through the implementation of the ``Death in Custody Reporting Act'' of 2013 (DCRA, P.L. 113-242). Under this act, states are required to provide data regarding the death of any person related to policing activity (detained, under arrest, in the process of being arrested, etc.). Starting January 2019, also the Federal Bureau of Investigation(FBI) has started collecting data within the ``National Use-of-Force Data Collection'' program.\footnote{\url{https://crime-data-explorer.app.cloud.gov/pages/home}}. This program is open to all federal, state, local, and tribal law enforcement and investigative agencies, although participation to it is not enforced, rather on voluntary basis. 

However, to the day,  policymakers are fundamentally limited by the insufficient amount, quality and accessibility of policing data. And, more generally, the data-scarcity has prevented the wider scientific community from conducting thorough quantitative analysis on police activity and behavior (see, e.g.\ \citet{peeples2019data,peeples2020data} for recent overviews of the problem of the lack of data in the context of police brutality). As a consequence, we are still far from having developed effective ``data-driven'' policy-making pipelines. Indeed, recent work has investigated how the incompleteness and bias of available data can lead to unwarranted data-driven solutions: \citet{richardson2019dirty} discuss how developing predictive policing system which leverage ``bad'' historical data --- corrupted incomplete and racially biased --- can produce dangerous downstream predictions, potentially violating individual civil rights. Relatedly, machine learning scientists have started investigating the problem of ``algorithmic fairness'', analyzing  risks and flaws related to the employment of black-box, automated decision systems within complex data-driven policy making procedures \citep{veale2018fairness, sloane2019ai, d2020fairness}. Because of all these reasons, it is imperative for the community to collect, curate and provide freely-accessible, easy-to-use data on policing activity.


In the present work, we provide a new dataset that contains information on the personnel, activities, use of force, and complaints in the Chicago Police Department (CPD). \textcolor{red}{maybe short discussion of why Chicago is ``special''? + cite some work on policing activity in Chicago? + cite some relevant recent work eg \citep{ba2021role}}. The original data comprises a collection of datasets, each obtained following different requests covered by the Freedom of Information Act (FOIA) to the Chicago Police Department (CPD) and the Civilian Office of Police Accountability (COPA) on the CPD's personnel and its activities. The data on complaints against police include complaints filed by citizens or internally by other members of the department.  Prior to our work, joint, coherent use of these different dataset for data-analysis purposes was highly non-trivial: for example, publicly available data does not contain coherent unique identifiers for officers across the different datasets. We discuss in detail in \Cref{sec:data} our data cleaning, linking and formatting process. Next, we provide examples and use-cases on the datasets in \Cref{sec:analysis}. These include, but are not limited to, prediction tasks (e.g., predicting officer's misconduct on the basis of their traits, experience, and assigned units), network analysis (e.g., detecting communities within the social network of officers co-listed on complaints), spatiotemporal data analysis (e.g., investigating patterns of officer shooting events), causal inference (e.g., tracking the effects of new disciplinary practices, new training techniques, and new oversight on complaints and use of force). We conclude with a discussion of intended use, and future research directions in \Cref{sec:discussion}.


%%%%%% DARIA'S TEXT

%
%There has been some ML work done on individual correlates that are associated
%with (and potentially predictive of) police misconduct. Scholars have looked at
%the association between misconduct and an individual officer's race, gender,
%years on the force, prior instances of misconduct, stages of officer career,
%assigned unit. But research suggests that violence emerges not just at the
%individual level but also at the population level from the interaction of
%individuals in groups—think of gangs. 
%
% 
%We offer a dataset that allows ML to investigate the influence of police social
%networks on police misconduct. The dataset allows ML to explore features of
%police social networks that might be associated with and potentially predictive
%of adverse police encounters with civilians or other misconduct. These network
%correlates can include the architecture and structures of social networks in
%departments and smaller subunits within a department—for example, the number of
%connections each officer has, the strength of those connections, the bridge
%positions that some officers have across subunits. More innovatively, ML can be
%used in the way it is often used to study disease dynamics: ML can find the
%patterns of behavior (clustering, outbreaks, spreading, geographic hotspots) on
%a network that might be associated with violence. 
%
% 
%We show how to use complaint datasets from a police department in a major city
%to investigate the influence of social networks on police misconduct and
%violence. 
%
%
%The original data was obtained following a series of requests covered by the
%Freedom of Information Act (FOIA) to the Chicago Police Department (CPD) and
%the Civilian Office of Police Accountability (COPA). The information which was
%requested pertained to CPD's personnel and its activities.
%
%\tbo{merge}
%The data on complaints against police include complaints filed by citizens or internally by other members of the department. This data was obtained by Jamie Kalven, an independent journalist represented by the University of Chicago’s Mandel Legal Aid Clinic, who filed Illinois Freedom of Information Act requests with the Chicago Police Department. These FOIA requests asked for documents containing the names of repeat police complainees, lists that had been produced in discovery in lawsuits alleging police abuse. In Kalven v. City of Chicago, an Illinois appellate court issued a general ruling in March of 2014 that documents bearing on allegations of police abuse are public information. Following the decision, the non-profit Invisible Institute began to collaborate with Kalven and the Mandel Legal Aid Clinic to follow up on earlier FOIA requests and to file new ones. The data disclosed in response to these earlier and now ongoing FOIA requests are uploaded on the Invisible Institute’s website and are made publicly available. 
%
%\textbf{TODO:} ideally say a bit more about the history of these FOIA requests.
%Apparently they were initiated by individual journalists and lawyers and were
%later coordinated by the Invisible Institute which ultimately became the
%central location were (almost) all the data is currently available.
%
%\begin{table}[h]
%	\begin{center}
%\begin{tabular}{@{}llll@{}}
%	\toprule
%	request \#&received&requested&description\\
%\midrule
%	\texttt{P0-58155}&2017-04-17& &Officer roster\\
%	\texttt{P4-41436}&2018-03-21& &Officer roster\\
%		\texttt{P0-52262}&2016-12-04&2016-09-19&Unit assignment\\
%		\texttt{16-1105}&2016-03-11&2016-02-10&Unit assignment\\
%	\texttt{P0-46957}&2016-06-29&2016-04-22&Complaints (CPD)\\
%	\texttt{18-060-425}&2018-08-28&2018-08-20&Complaints (COPA)\\
%	\texttt{P0-46360}& & &Tactical Response Reports\\
%\bottomrule
%\end{tabular}
%\caption{Summary of the FOIA requests to the CPD and COPA contained in our repository.}
%\label{table:summary}
%\end{center}
%\end{table}
%
%Each FOIA request is identified by a request number, \cref{table:summary} gives
%an overview of all the requests made to the CPD and COPA that are present in
%our repository. This information is also available in the file
%\texttt{dataset.csv} in the root folder of the repository. The original
%data—that is the files received after each FOIA request—are present in the
%\texttt{raw/} folder of the repository, with one subfolder for each request,
%identified by the request number. When available, each subfolder also contains
%the formal request letter as well as the reply letter from the CPD or COPA,
%which are useful in understanding what data was included in each dataset.
%Some additional comments about the data:
%\begin{itemize}
%	\item \emph{Officer roster:} lists all officers (past or present) employed
%		by the CPD along with attributes such as year of birth, age, race,
%		gender, appointment date, resignation date, etc.
%	\item \emph{Unit assignment:} the CPD is organized into (500 or so?) units.
%		Each officer can be assigned to one or multiple units and these
%		assignments can change over time. The unit assignment datasets contain
%		one record for each officer and each unit they were assigned to, with
%		the start date and end date of this assignment.
%	\item \emph{Complaints:} formal complaints filed by citizens against police
%		officers. Complaints are identified by a complaint number, and there is
%		one record for each complaint and each officer listed on the complaint,
%		indicating the allegation made against them, result of the
%		investigation of the allegation (with possible sanction), etc.
%	\item \emph{Tactical Response Reports:} these are forms that officers are
%		required to file after each incident for which the officer's response
%		involved use of force.
%\end{itemize}
%
%Let us already mention an inherent difficulty in making use of this data, which
%will be discussed in \cref{sec:linking}: there is no number/identifier which
%uniquely identifies officers across datasets. Such an identifier probably
%exists internally in the CPD, but was never included in the data released to
%the public. One would be tempted to believe that the \emph{badge number} (also
%sometimes referred to as \emph{star number} of an officer is such an
%identifier, but unfortunately, it changes over the course of an officer's
%career in the CPD, and a given badge number can be reassigned to different
%officers when they are no longer in use.
%
%
%\subsection{Related Work}
%
%Police records datasets
%
%This data was originally obtained by the Invisible Institute and has been publicly available for about 3 years here \url{https://github.com/invinst/chicago-police-data}
%(limitations: not well organized or reproducible, problematic/incorrect linkage, missing unit information, units are not semantically grouped)
%
%Gang violence data


\section{Related Work}\label{sec:related}

Existing public datasets on policing are extremely limited. All police
departments collect data on police personnel, activities, operations and
complaints, but few agencies disclose such data to the public \cite{Jackman21}.
A small but increasing number of departments are using the data to develop
machine learning approaches to “early intervention” warning systems that
predict adverse encounters between civilians and police. However, the datasets
used for these “early intervention systems” remain confidential, even when
scholars publish the results of their work on such systems \cite{Helsby18}.

In response to public pressure on the issue of police violence, a number of
public datasets on policing have recently emerged. These datasets, which are
generated by both private and public actors, are hard to reproduce and often
incomplete. For example, the Police Data Initiative (sponsored by a non-profit
on science in policing) is a collection of datasets from more than 130 state
and local law enforcement. These datasets often contain raw, uncleaned data,
and almost never include information about individual officer behavior
\cite{pdi}.  Stanford University’s Open Policing Initiative, which contains
data on traffic stops from around the country, does not include information
about identifiable individual officers \cite{sopp}.

The NYPD Misconduct Complaint Database, which has been compiled by the New York
Civil Liberties Union, includes complaint data made available by the New York
Police Dept.\ Civilian Complaint Review Board (a civilian oversight
organization.) This dataset includes information identifying officers by name
and is searchable as well, but the dataset includes only uncleaned data.
 
Current data collection efforts by government agencies can overcome some of
these limits, but can also face political and legal challenges. Owing to
political resistance by the Trump Administration, the Department of Justice
(DOJ) only began last year to collect data on arrest-related deaths through the
implementation of the ``Death in Custody Reporting Act'' of 2013. Under this
act, states are required to provide data regarding the death of any person
related to policing activity (detained, under arrest, in the process of being
arrested, etc.) But states often cannot compel local law enforcement to
disclose the data in the absence of state legislation. Likewise, the Federal
Bureau of Investigation (FBI) ``National Use-of-Force Data Collection'' program
is voluntary and not mandatory.

Policing scholars have compiled their own datasets. In a paper on the impact of
race and gender on policing, economist Bocar Ba and his research team generated
a publicly available dataset with information on individual officers in the
Chicago Police Department. This dataset linked officer demographic information
with their arrests, stops, assignment history, and data on the districts to
which the officers were assigned. Documentation for this dataset was
appropriate but sparse, making the reproduction of the dataset from scratch
somewhat challenging.

Other scholars have drawn on the same Invisible Institute data to create
a dataset similar to ours, but with far less documentation and focus on
reproducibility. In a paper on the impact of complaints on misconduct, Legal
scholars Rozema and Schanzenbach used the Invisible Institute raw data files to
link information on officer complaints with litigation settlements involving
the officers. However, neither the paper nor the publicly available dataset
provided any information about cleaning, processing or linking the data. 


\section{The CPD Data} \label{sec:data}
The original raw data released by the CPD, 
as well as the code to clean the data and generate this document, 
are both available at \url{https://github.com/chicago-police-violence/data}.

\subsection{Data origin}
This data was obtained by Jamie Kalven, an independent journalist represented
by the University of Chicago’s Mandel Legal Aid Clinic, who filed Illinois
Freedom of Information Act requests with the Chicago Police Department. These
FOIA requests asked for documents containing the names of repeat police
complainees, lists that had been produced in discovery in lawsuits alleging
police abuse. In Kalven v. City of Chicago, an Illinois appellate court issued
a general ruling in March of 2014 that documents bearing on allegations of
police abuse are public information. Following the decision, the non-profit
Invisible Institute began to collaborate with Kalven and the Mandel Legal Aid
Clinic to follow up on earlier FOIA requests and to file new ones. The data
disclosed in response to these earlier and now ongoing FOIA requests are
uploaded on the Invisible Institute’s website and are made publicly available. 

\textbf{TODO:} ideally say a bit more about the history of these FOIA requests.
Apparently they were initiated by individual journalists and lawyers and were
later coordinated by the Invisible Institute which ultimately became the
central location were (almost) all the data is currently available.




\begin{table}[h]
	\begin{center}
\begin{tabular}{@{}lllll@{}}
	\toprule
	request \#&received&requested&description\\
\midrule
	\texttt{P0-58155}&2017-04-17& &Officer roster\\
	\texttt{P4-41436}&2018-03-21& &Officer roster\\
	\texttt{P0-52262}&2016-12-04&2016-09-19&Unit assignment\\
	\texttt{16-1105}&2016-03-11&2016-02-10&Unit assignment\\
	\texttt{P0-46957}&2016-06-29&2016-04-22&Complaints (CPD)\\
	\texttt{18-060-425}&2018-08-28&2018-08-20&Complaints (COPA)\\
	\texttt{P0-46360}& & &Tactical Response Reports\\
	\texttt{P0-46987}&2016-05-13&2016-04-25&Unit names\\
	\texttt{P0-61715}& &2017-07-26&Awards\\
	\texttt{P5-06887}&2019-10-11&2019-07-19&Awards\\
	&2017-09-27&2017-09-13&Salary\\
\bottomrule
\end{tabular}
\caption{Summary of the FOIA requests contained in our repository (blanks are missing entries).}
\label{table:summary}
\end{center}
\end{table}



Each FOIA request is identified by a request number, \cref{table:summary} gives
an overview of all the requests made to the CPD and COPA that are present in
our repository. This information is also available in the file
\texttt{dataset.csv} in the root folder of the repository. The original
data—that is the files received after each FOIA request—are present in the
\texttt{raw/} folder of the repository, with one subfolder for each request,
identified by the request number. When available, each subfolder also contains
the formal request letter as well as the reply letter from the CPD or COPA,
which are useful in understanding what data was included in each dataset.
Some additional comments about the data:
\begin{itemize}
	\item \emph{Officer roster:} lists all officers (past or present) employed
		by the CPD along with attributes such as year of birth, age, race,
		gender, appointment date, resignation date, etc.
	\item \emph{Unit assignment:} the CPD is organized into (500 or so?) units.
		Each officer can be assigned to one or multiple units and these
		assignments can change over time. The unit assignment datasets contain
		one record for each officer and each unit they were assigned to, with
		the start date and end date of this assignment.
	\item \emph{Complaints:} formal complaints filed by citizens against police
		officers. Complaints are identified by a complaint number, and there is
		one record for each complaint and each officer listed on the complaint,
		indicating the allegation made against them, result of the
		investigation of the allegation (with possible sanction), etc.
	\item \emph{Tactical Response Reports:} these are forms that officers are
		required to file after each incident for which the officer's response
		involved use of force.
\end{itemize}


\paragraph{Technical choices.} Code is written in Python, use of \texttt{Make}
to coordinate the various processing steps and easily reproduce them.
\textbf{TODO:} say a bit more about the requirements of the environment, or
maybe simply refer to the README?



\subsection{Initial Cleaning}

The files initially released by the CPD as a reply to the FOIA requests are for
the most part Excel spreadsheets, with inconsistent formatting and which can
thus be difficult to process programmatically. As an example, the reader is
invited to open the file
\texttt{p046957\_-\_report\_1.1\_-\_all\_complaints\_in\_time\_frame.xls}
available in the folder \texttt{raw/P0-46957/}. As can be seen, each record in
this file is spread over two rows of the spreadsheet, with the field names
repeated at the beginning of the second row for each record.

The goal of the cleaning step is thus to produce ``reasonable'' CSV files from
the original files, with the minimum requirement that each record be presented
on a single line after this step. The code for this cleaning step is contained
in the files \texttt{datasets.py}, \texttt{parse.py} and
\texttt{parse\_p046957.py} in the \texttt{src/} folder and the entire step can
be applied by running \texttt{make parse} in the root of the repository. This
creates the folder \texttt{parsed/} containing the clean CSV files.

As can be seen by inspecting the code, the decisions made at this stage are, we
believe, uncontroversial as they only consists of:
\begin{itemize}
	\item unifying field names across datasets, so that the same type of data
		is always identified in the saw way (for example, \texttt{Appointment
		Date}, \texttt{Appt Date}, \texttt{appointment\_date} are all mapped to
		\texttt{appointment\_date}.
	\item unifying field values across datasets. For example, the gender of an
		officer is indicated as a single letter \texttt{M} or \texttt{F} in
		some datasets or as \texttt{Male}, \texttt{Female} (based on the data
		release, it does not seem that the system used by the CPD has an option
		to represent non-binary officers).
	\item parsing values into the correct data type or format. For example,
		dates are formatted differently depending on the dataset, and we map
		everything to the ISO 8601 format.
\end{itemize}

Consequently, for someone planning to use the data in the present repository,
there is virtually no reason not to start at the minimum from the output of
this cleaning step. The subsequent steps required making more difficult and
debatable decisions, so depending on the application, researchers might want to
perform them differently, but in all cases, those alternative decisions can
branch off from the output of the cleaning step.

\subsection{Linking and Merging Datasets}\label{sec:linking}

\subsubsection{Officer matching}

As already alluded to, the main challenge at this step is that there is no
identifier uniquely identifying officers across records. In other words, there
is no foolproof way to know if two different records correspond to the same
officer, \emph{even within the same dataset} (for example the same officer
could appear with slightly different attributes on two different complaint
records).

We thus need to design a matching method striking a balance between
\begin{itemize}
	\item being loose enough to avoid type II error (false negatives). If the
		same officer appears with slightly different attributes across two
		records, we do not want our matching method to believe it is two
		different officers.
	\item being strict enough to avoid type I error (false positives). We do
		not want to merge two different officers into a single identity.
\end{itemize}

The difficulty in achieving this balance is that perhaps surprisingly
\emph{none of the attributes of a given police officer are guaranteed to be
stable over time}. Most notably, officers' names change over time, for example
to fix data entry errors or in case of legal name changes. However, we observed
two attributes, present in almost all original datasets and which seem
remarkably stable over the time: \emph{appointment date} and \emph{birthyear}.
These two attributes thus proved very valuable to disambiguate officers with
identical names.

In order to match officers across two datasets, we developed an \emph{iterative
pairwise matching procedure} that makes iterative passes over the datasets.
During each pass, a subset of the officer attributes present in both datasets
is selected as the matching criterion, and a pair of officers (one from each
dataset) is identified as a \emph{match} if (i) their attributes from the
chosen subset match, and (ii) if they are the only two officers matching on
these attributes. Once a pair is identified as a match, it is put aside, and
the next pass is performed on the remaining unmatched officers. After all the
passes are done the leftover officers are declared as different officers. By
constructing a hash table mapping a subset of attributes to the list of
officers sharing these attributes, each pass can be performed in linear time,
so the overall running time of the procedure is $O\big(P(N_1+N_2)\big)$ where
$P$ is the number of passes and $N_1, N_2$ are the number of officers in each
dataset.

To fully specify the matching procedure we thus need to specify which subset of
attributes is chosen as the matching criterion at each pass. For this, we go
from the stricter to the looser criterion: for the first pass, we choose
a subset \emph{all} the attributes which are present in two datasets, and then
start removing attributes one by one. For example, one can remove the
\emph{last name} attribute for the second pass to match a pair of officers
whose last names are different but match on all the remaining attributes, thus
identifying an officer whose last name changed between the releases of the two
datasets. The advantage of going from stricter to looser is two fold:
\begin{itemize}
	\item starting from the strictest set of attributes identifies the
		\emph{unambiguously matching pairs}, that is, the ones which are
		a clear match and which, thankfully, constitute the vast majority of
		officers (typically around 80-90\% of officers are matched during the
		first pass \textbf{TODO check}). Since the next passes will only be
		performed on the remaining unmatched officers, this removes a lot of
		potential ambiguities which could occur once the set of attributes is
		reduced.
	\item since the vast majority of officers is matched during the first pass,
		it becomes feasible during the subsequent passes to visually inspect all
		the matched pairs and assess whether the chosen set of attributes was
		too strict or too lose (\textbf{TODO:} explain how to activate
		debugging information in the code).
\end{itemize}

We note that there is still some amount of subjective judgment involved,
following a visual inspection of the second and subsequent passes, to decide
which sets attributes are ``acceptable'' (that is, for which the probability of
two persons sharing these attributes in a population of the size of the CPD is
extremely small). This is also how we decide that sufficiently many passes have
been performed: when it would seem likely to introduce a type I error by
matching any of the remaining officers. As a general rule of thumb we erred on
the side of favoring type II errors over type I errors. That is, we only
matched officers when it would seem extremely unlikely that they correspond to
two different individuals).

\textbf{TODO:} table of officer attributes in each dataset

\textbf{TODO:} explain the few subtleties where we don't use equality to match
attributes (for example when matching age and birthyear, where the age only
lets us identify the birthyear with an accuracy of 1 year). Or with stars,
where we test whether a star number is contained in the subset of known stars
for this officer.

With this procedure at hand we can thus link officers across datasets starting
from the most similar datasets first (for which we expect to have the least
amount of ambiguity). That is we first link \texttt{P0-58155} to
\texttt{P4-41436}, then \texttt{P0-52262} to \texttt{16-1105} and then the
remaining datasets \textbf{expand}

\textbf{TODO} give in appendix table summarizing which subset of attributes are
used at each linking operation 

\subsubsection{Roster consolidation}

A byproduct of matching officers across datasets is that for each officer, we
now have as many “profiles” as the number of datasets in which they appear,
where by \emph{profile} we mean a collection of attributes. Note that each
profile can contain a different subset of attributes (since not all attributes
are present in each dataset) and that a given attribute might take a different
value in different profiles of the same officer (since the iterative matching
procedure is not restricted to performing strict matching).

We are thus faced with the task of “consolidating” the different profiles of
a given officer into a single profile. Of course, if an attribute is present in
a single profile, and absent from the others, this is the value we keep in the
consolidated profile. But if an attribute appears with different values across
different profiles, we choose the value coming from the profile corresponding
to the \emph{most recent data release}.

\subsubsection{Unit assignment history} TODO.


\subsection{Unit identification and binning}


% !TEX root = ../main.tex

\section{Examples and data analysis} \label{sec:analysis}

In this section, we provide examples and use-cases on the dataset. Code to reproduce our examples is available online, in the \texttt{examples} folder.

\subsection{Summary statistics}

First, we start by providing some summary statistics. 

\subsubsection{The roster file}
The file \texttt{roster.csv} contains information about $N=35{,}430$ CPD officers. For each officer --- identified by a Unique Identifier (``uid''), \texttt{roster.csv} provides, among other covariates, the officer's name, gender, race, birthyear, appointment and resignation dates. It is straightforward  to use the data to generate summary statistics, e.g., using appointment and resignation dates to understand the number of appointments, retirements as well as active officers over the years of  (\Cref{fig:history}). We report in \Cref{tab:stats} some 
\begin{figure}[h] 
	\includegraphics[width=\textwidth]{figs/history} 
	\caption{Historical data from the CPD. The left subplot shows the number of officers' appointments, the central subplot the number of officers' resignations, while the right subplot shows the number of active officers in the database (vertical axis) as a function of time (horizontal axis).} \label{fig:history}
\end{figure}

\begin{table}[h]
\begin{tabular}{l|c|c|c|c|c|c|c|c|c|}
\cline{2-3} \cline{5-10}
                                               & \multicolumn{2}{c|}{\textit{\textbf{Gender}}} & \multicolumn{1}{l|}{} & \multicolumn{6}{c|}{\textit{\textbf{CPD Race Category}}}                                                                                                                                                   \\ \cline{2-3} \cline{5-10} 
                                               & \textit{\textbf{M}}   & \textit{\textbf{F}}   &                       & \textit{\textbf{White}} & \textit{\textbf{Black}} & \multicolumn{1}{l|}{\textit{\textbf{Hisp.}}} & \textit{\textbf{Asian}} & \multicolumn{1}{l|}{\textit{\textbf{Am. Ind.}}} & \textit{\textbf{Bl. Hisp.}} \\ \cline{1-3} \cline{5-10} 
\multicolumn{1}{|c|}{\textit{\textbf{All}}}    & 28316                 & 7122                  &                       & 21047                   & 8599                    & 4811                                         & 582                     & 67                                              & 9                           \\ \cline{1-3} \cline{5-10} 
\multicolumn{1}{|c|}{\textit{\textbf{Active}}} & 11118                 & 4452                  &                       & 7241                    & 3895                    & 3596                                         & 467                     & 40                                              & 9                           \\ \cline{1-3} \cline{5-10} 
\end{tabular}
\caption{Summary statistics (gender, race) for all officer (first row), as well as for active officers --- those whose resignation date is not before Jan 1st 2019.} \label{tab:stats}
\end{table}

\begin{figure}[h] 
	\includegraphics[width=\textwidth]{figs/history_by} 
	\caption{Historical data from the CPD. Birthyears for officers in the CPD dataset: blue dots count all officers, while red count only officers who are ``active'' as of January 1st, 2019.} \label{fig:history_by}
\end{figure}

\subsection{Tactical response reports}

The file \texttt{tactical\_response\_reports.csv} contains information about tactical response reports, filed as a consequence of  events involving use of force of CPD officers. This dataset contains $9{,}246$ distinct events, as identified by the index ``\texttt{event\_no}'', between January 17th 2004 and April 12th 2016. For each event, the time and location is recorded. Details about the officers involved, as well as the civilian subject are available too. 

\begin{figure}[h] 
	\includegraphics[width=\textwidth]{figs/trrs_times} 
	\caption{Temporal data for TRRs.} \label{fig:trrs_times}
\end{figure}

\begin{figure}[h] 
	\includegraphics[width=\textwidth]{figs/trr_stats} 
	\caption{Temporal data for TRRs.} \label{fig:trrs_stats1}
\end{figure}

\begin{figure}[h] 
	\includegraphics[width=\textwidth]{figs/trr_stats_race_race} 
	\caption{Temporal data for TRRs.} \label{fig:trrs_stats2}
\end{figure}

\subsection{Complaints}

\begin{figure}[h] 
	\includegraphics[width=\textwidth]{figs/complaints_times} 
	\caption{Temporal data for complaints.} \label{fig:complaints}
\end{figure}
\subsection{Salary}

\begin{figure}[h] 
\includegraphics[width=\textwidth]{figs/salary} 
\caption{Historical data from the CPD. Salary versus experience in each
position, years x axis, salary (USD) y axis, for a few of the most common
positions. Lines are means, bars indicate 1 std dev above and below.} \label{fig:salary}
\end{figure}

\begin{figure}[h] 
\includegraphics[width=\textwidth]{figs/position_race} 
\caption{Historical data from the CPD. Fraction of officers in representative positions 
per race category.} \label{fig:salary}
\end{figure}

\begin{figure}[h] 
\includegraphics[width=\textwidth]{figs/position_gender} 
\caption{Historical data from the CPD. Fraction of officers in representative positions 
per gender category.} \label{fig:salary}
\end{figure}

\subsection{Awards}

\begin{figure}[h] 
\includegraphics[width=\textwidth]{figs/awards} 
\caption{Historical data from the CPD. Awards per officer vs race for the two cpd gender categories.} \label{fig:awards}
\end{figure}




\section{Discussion}

\subsection{Intended Use}
Benchmark algorithms for:
\begin{itemize}
	\item community detection (where for example unit assignment can be used to
		have a ground truth
	\item network inference from contagion data (complaint network can be used
		as ground truth, shootings as the contagion)
	\item time series prediction over networks
\end{itemize}

Relatedly, useful to assess and design contagion models


\textbf{TODO:} the following was copy-pasted from Daria's email.

In my view, in discussing intended uses for the data set, we should probably focus attention at the outset on the benchmarking capacities for our data:

\begin{itemize}
	\item Algorithms to detect network spatial and temporal clustering
	\item Algorithms to fit models of contagion on the network (in addition to shooting generally, researchers can examine other behaviors independently documented, like foot-chase shootings, use of force during arrest, abusive charging behavior, but also laudatory behavior like complying with reporting requirements, etc.)
\end{itemize}

Other uses for this dataset can include:

\begin{itemize}
	\item Correlating police behavior to police traits and complainant traits (like race, gender, location of assignment/residence, years on force, etc.)
	\item Correlating complaint filing frequency to police traits and complainant traits (see above)
	\item Creating network graphs for subunits of interest (particular police patrol districts, for example).
	\item Tracking complaint trends relative to external events like new disciplinary practices, new training techniques, new oversight (Department of Justice, police union monitoring, new civilian oversight, publication of shootings in major papers) and other events like high-profile scandals, introduction of new technologies like Tasers, etc.
	\item Subject to caveats, investigating how to control for the dangerousness of police work when investigating the frequency of particular police behaviors (use of excessive force, wrongful arrest, abusive speech, etc.)
\end{itemize}

\subsection{Limitations}

todo


\section*{Acknowledgments}
T. Campbell was supported by a National Sciences and Engineering Research Council of Canada (NSERC) Discovery Grant and an NSERC Discovery Launch Supplement.

\bibliographystyle{unsrtnat}
\bibliography{references}

\input{sections/checklist}

\appendix

\section{Datasheet}

\subsection{Motivation}

\paragraph{For what purpose was the dataset created?}

\paragraph{Who created the dataset, and on behalf of which entity?}

\paragraph{Who funded the creation of the dataset?}

\paragraph{Any other comments?}

\subsection{Composition}

\paragraph{What do the instances that comprise the dataset represent?}

\paragraph{How many instances are there in total (of each type)?}

\paragraph{Does the dataset contain all possible instances or is it a sample of instances from a larger set?}

\paragraph{What data does each instance consist of?}

\paragraph{Is there a label or target associated with each instance?}

\paragraph{Is any information missing from individual instances?}

\paragraph{Are relationships between individual instances made explicit?}

\paragraph{Are there recommended data splits?}

\paragraph{Are there any errors, sources of noise, or redundancies in the dataset?}

\paragraph{Is the dataset self-contained, or does it rely on external resources?}

\paragraph{Does the dataset contain data that might be considered confidential?}

\paragraph{Does the dataset contain data that, if viewed directly, might be offensive, insulting, or threatening?}

\paragraph{Does the dataset relate to people?} 

\paragraph{Does the dataset identify any subpopulations?}

\paragraph{Is it possible to identify individuals?}

\paragraph{Does the dataset contain data that might be considered sensitive in any way?}

\subsection{Collection Process}

\paragraph{How was the data associated with each instance acquired?}

\paragraph{What mechanisms or procedures were used to collect the data?}

\paragraph{If the data are a sample from a larger set, what was the sampling strategy?}

\paragraph{Who was involved in the data collection process and how were they compensated?}

\paragraph{Over what timeframe was the data collected?}

\paragraph{Were any ethical review processes conducted?}

\paragraph{Does the dataset relate to people?}

\paragraph{Did you collect the data from the individuals directly, or obtain it via third parties?}

\paragraph{Were the individuals notified about the data collection?}

\paragraph{Did the individuals in question consent to the collection and use of their data?}

\paragraph{If consent was obtained, were the consenting individuals provided with a mechanism to revoke their consent in the future or for certain uses?}

\paragraph{Has analysis of the potential impact of the dataset and its use on data subjects been conducted?}

\subsection{Preprocessing and cleaning}

\paragraph{Was any preprocessing of the data done?}

\paragraph{Was the ``raw'' data saved in addition to the cleaned data?}

\paragraph{Is the software used to clean the data available?}

\subsection{Uses}

\paragraph{Has the dataset been used for any tasks already?}

\paragraph{Is there a repository that links to any or all papers that use the dataset?}

\paragraph{What (other) tasks could the dataset be used for?}

\paragraph{Is there anything about the composition of the dataset or the way it was collected and cleaned that might impact future uses?}

\paragraph{Are there tasks for which the dataset should not be used?}

\subsection{Distribution}

\paragraph{Will the dataset be distributed to third parties outside of the entity on behalf of which the dataset was created?}

Yes, the data is publicly available.

\paragraph{How will the dataset be distributed?}

It is available on GitHub at \url{https://github.com/chicago-police-violence/data}.

\paragraph{When will the dataset be distributed?}

It is currently publicly accessible.

\paragraph{Will the dataset be distributed under a copyright, other IP license, or terms of use?}

Yes; the source code is released under the MIT license, and the data output by the cleaning code is released under the Creative Commons 4.0 BY-NC-SA license.

\paragraph{Have any third parties imposed IP-based or other restrictions on the data associated with the instances?}

No.

\paragraph{Do any export controls or other regulatory restrictions apply to the data?}

No.

\subsection{Maintenance}

\paragraph{Who is supporting/hosting/maintaining the dataset?}

The repository will be hosted on GitHub. As of August 2021, the repository
owners are Thibaut Horel, Trevor Campbell, and Lorenzo Masoero, but ownership
may change over time.

\paragraph{How can the data owner/curator be contacted?}

Issue threads on GitHub are the primary channel of contact for the repository maintainers.

\paragraph{Is there an erratum?}

Not as of yet. For each major release version, notes will be included and
hosted in the repository that will detail cleaning/linking errors that have been fixed.

\paragraph{Will the dataset be updated?}

The original raw source data from FOIA requests will not be modified. More raw
data files may be added over time corresponding to new FOIA requests. The data
cleaning and linking code will be edited over time to fix errors; release
versions will be clearly marked on GitHub.

\paragraph{If the dataset relates to people, are there applicable limits on the retention of data associated with the instances?}

No; this data was released per FOIA requests and is in the public domain.

\paragraph{Will older versions of the dataset continue to be supported/hosted/maintained?}

Yes; a full version-controlled history of the project will exist on GitHub.

\paragraph{If others want to extend/augment/build on/contribute to the dataset, is there a mechanism for them to do so?}

Yes; the repository for the dataset is hosted on GitHub, where pull requests are a usual channel for external contribution.

\section{Data Cleaning and Linking: Additional Details}\label{sec:app-cleaning}

This appendix contains an expanded version of \cref{sec:cleaning}.

\subsection{Initial Cleaning}

The files initially released by the CPD are for
the most part Excel spreadsheets, with inconsistent formatting and which can
thus be difficult to process programmatically. As an example, the reader is
invited to open the file
\texttt{p046957\_-\_report\_1.1\_-\_all\_complaints\_in\_time\_frame.xls}
available in the folder \texttt{raw/P0-46957/}. As can be seen, each record in
this file is spread over two rows of the spreadsheet, with the field names
repeated at the beginning of the second row for each record.

Consequently, the goal of the cleaning step is to produce uniformly formatted
CSV files, with the minimum requirement that each record be presented on
a single line after this step. The code is contained in the files
\texttt{datasets.py}, \texttt{parse.py}, \texttt{parse\_p046957.py}, and \texttt{parse\_p061715.py} in the
\texttt{src/} folder.  The step can be applied by running \texttt{make prepare}
in the root of the repository, which creates the folder \texttt{tidy/}
containing the cleaned CSV files.

The decisions made at this stage are straightforward and involve no subjective
judgment. They consist of:
\begin{itemize}
	\item Unifying attribute names across datasets, so that the same type of
		data is always identified in the same way (for example,
		\texttt{Appointment Date}, \texttt{Appt Date},
		\texttt{appointment\_date} are all mapped to
		\texttt{appointment\_date}).
	\item Unifying attribute values across datasets. For example, the gender of
		an officer is indicated as a single letter \texttt{M}/\texttt{F} in
		some datasets, and as \texttt{Male}/\texttt{Female} in others. A similar issue
		arises with the race of officers, sometimes given as a three-letter
		code, and sometimes described in full. In such cases we map all
		possible forms of an attribute value to a canonical one.
	\item Parsing values into the correct data type or format. For example,
		dates and times are formatted differently depending on the dataset, and
		we map everything to the ISO\,8601 format. Integers are also parsed so
		as to remove the various paddings present in the original data.
	\item Concatenating all the files containing a given type of record in each
		data release. Indeed, the CPD's responses to some FOIA requests split
		the records chronologically over multiple Excel spreadsheets (see for
		example \texttt{raw/P0-46957}) or multiple tables within spreadsheets
                (see for example \texttt{raw/salary}). Having a single file for each record
		type instead simplifies later processing steps.
\end{itemize}

Although this initial cleaning is only the first step in the process producing
our final dataset, we structured our processing code in such a way that it is
easy to stop the process at this step and keep the intermediate output in the
\texttt{tidy/} folder. This is because the next steps required making more
subjective judgment calls, e.g., to decide how to clean erroneous records and resolve
ambiguities arising from the merging and linking of datasets. Consequently, it
is possible that some applications will require performing these next steps
differently. In such cases, the files contained in the \texttt{tidy/} folder
should provide a safe intermediate point at which to branch off from our
processing pipeline.

\subsection{Linking and Merging Datasets}\label{sec:linking}

\paragraph{Overall description.}

As discussed in \cref{sec:raw}, the main challenge in processing the raw data
is that officers are not uniquely identified across datasets. In other words,
there is no foolproof way to know if two different records from two different
datasets correspond to the same individual. For example, a na\"ive matching
procedure identifying two officers as being the same individual whenever they
have the same name is inadequate: given the size of the CPD roster (over 35\,000
active or retired officers), it is guaranteed to contain many homonymous
individuals. Furthermore, even though the original datasets contain several
identifying attributes (name, birth year, appointment date, race, gender),
those can change over item as mentioned in \cref{sec:raw}.

At a high level, our procedure \emph{grows} a population of uniquely identified
officers by sequentially examining each FOIA release. Each unique officer in
this population is assigned a unique identification (UID), which is a random
hexadecimal string (e.g.\ \texttt{9bc51eef-c37b-4eff-a14d-7e69f56b3d1e}), and
to each UID is associated a list of \emph{officer profiles}. There is one
\emph{officer profile} for each unique officer and each FOIA release in which
this officer appeared, listing the identifying attributes of this officer as
they appear in this given data release. In detail, for each FOIA release:
\begin{enumerate}
	\item We build a list of all the \emph{officer profiles} appearing in this
		release.
	\item For each officer profile, we attempt to \emph{match} it against the
		profiles of the population of unique officers constructed so far from
		previously examined FOIA releases.
		\begin{itemize}
			\item If the match is successful, we have identified a unique officer in
				the population whose profiles unambiguously match with the
				current profile. In this case, we simply attach the current
				profile to this officer and UID, and
				the population does not grow.
			\item If the match is unsuccessful, we add a new officer with a new UID to the population of unique
				officers and attach the current profile to this new officer.
		\end{itemize}
\end{enumerate}

After all FOIA releases have been processed, the population contains the set of
all unique officers appearing in the original data. Each officer is represented
by UID and a collection of profiles, representing the various ways in which
this unique officer appears across different datasets. These profiles can be
found in the file \texttt{final/officer\_profiles.csv}.
At this point the \texttt{final/} folder also contains
one file for each type of record (complaints, tactical response reports, etc).
Within these files, officers are identified solely by their UID and other attributes
are removed; this avoids duplication of
information since these attributes are redundant with those found in
\texttt{final/officer\_profiles.csv}.

Note that this procedure depends on the order in which the FOIA releases are processed,
since each release will be matched against the profiles from all
previously considered releases. We chose to first process the two roster datasets
(\texttt{P0-58155} and \texttt{P4-41436}) since they are most similar and
supposed to contain one record for each officer in the CPD. Next, we process
the two unit assignments datasets (\texttt{16-1105} and \texttt{P0-52262})
since they are also supposed to cover the entire CPD. Finally, we process the
complaints, tactical response reports, awards, and salary data, in this
order.

\paragraph{Iterative pairwise matching.}
It remains to describe the \emph{match} operation which was left unspecified in
step 2.\ of the procedure above. This operation
matches a list of profiles found in the FOIA release being currently processed
against an already existing list of profiles and associated UIDs. We need to strike a balance between:
\begin{itemize}
	\item Being loose enough to avoid type II error (false negatives). If the
		same officer appears with slightly different attributes in two
		different profiles, we do not want the matching method to believe these
		profiles are attached to different officers.
	\item Being strict enough to avoid type I error (false positives). We do
		not want to attach a profile to an officer currently in the population
		if it in fact corresponds to a new officer.
\end{itemize}

We developed an \emph{iterative pairwise matching procedure}\footnote{This
procedure was inspired by a similar one developed by the Invisible Institute.},
that makes iterative passes over the list $L_1$ of profiles to be matched
against the list $L_2$ of already existing profiles. Recall that in our case,
the profiles in $L_2$ are associated with officers in the growing population of
unique officers, and hence they are each associated with a UID. During each
pass:
\begin{enumerate}
	\item A subset $S$ of the profile attributes is selected as the matching
		criterion.
	\item For each profile $p$ in $L_1$:
		\begin{enumerate}
			\item Construct the list $\ell_p$ of profiles in $L_2$ for which
				the attributes in $S$ match with $p$ exactly.
			\item If all the profiles in $\ell_p$ are associated with the same
				UID $u$, this is an unambiguous match and we can safely assign
				the UID $u$ to the profile $p$. We remove $p$ from $L_1$ and
				all the profiles associated with $u$ in $L_2$.
			\item Otherwise, the match is ambiguous and we keep $p$ in
				$L_1$.
		\end{enumerate}
\end{enumerate}
Observe that both $L_1$ and $L_2$ decrease in size as matches are found, so
each pass iterates over a smaller set of profiles than the previous one. When
all passes are done, the remaining profiles in $L_1$ are considered new
unique officers and assigned freshly generated UIDs. Furthermore, by
building at each pass a hash table mapping a subset of attributes to the
list of profiles in $L_2$ sharing these attributes, the construction of
$\ell_p$ in 2.(a) can be performed in constant time. The overall running time of the
  iterative matching procedure is $O\big(P(|L_1|+|L_2|)\big)$ where $P$ is the
  number of passes.\footnote{Running the iterative matching procedure on our
  largest dataset takes approximately 2 minutes on a standard personal laptop.}

Finally, to fully specify the matching procedure we need to describe how to
choose which subset $S$ of identifying attributes is chosen as the matching
criterion at step 1.\ during each pass. For this, we go from the stricter to
the looser criterion: in the first pass, $S$ contains \emph{all} the attributes
which are present in both $L_1$ and $L_2$, and then attributes are removed from
$S$ one by one in subsequent passes. For example, one can remove the \emph{last
name} attribute in the second pass to match a pair of profiles whose last names
are different but match on all the remaining attributes, thus identifying an
officer whose last name changed between two FOIA releases. The advantage of
going from stricter to looser is twofold:
\begin{itemize}
	\item Starting from the strictest set of attributes identifies the
		\emph{least ambiguous matches}, for which profiles match exactly on
		a large set of attributes and can confidently be considered to describe
		the same individual. This constitutes the vast majority of cases;
		more than 95\% of profiles are usually matched during the first pass.
		Since the next passes iterate only on the remaining profiles, this
		removes the majority of ambiguities that could arise as the
		matching criterion is relaxed.
	\item Since the vast majority of profiles are matched in the first pass,
		it becomes feasible in subsequent passes to visually inspect all
		the matched profiles and assess whether the chosen set of attributes was
		too strict or too lose.
\end{itemize}

We note that there is some amount of subjective judgement involved;
 visual inspection is employed in the second and subsequent passes to decide which sets
attributes are stringent enough as to avoid accidentally matching two different persons. 
This is also how we decide that sufficiently many passes have been
performed and that further relaxing the matching criterion would introduce too
many type I errors. In general, we erred on the side of favoring
type II errors over type I errors when uncertain.

%\textbf{TODO:} explain the few subtleties where we don't use equality to match
%attributes (for example when matching age and birthyear, where the age only
%lets us identify the birthyear with an accuracy of 1 year). Or with stars,
%where we test whether a star number is contained in the subset of known stars
%for this officer.

%\textbf{TODO:} table of officer attributes in each dataset


\subsection{Final cleaning and output}\label{sec:final-cleaning}

\paragraph{Roster consolidation.}
The procedure described in \cref{sec:linking} produces, for each unique officer
in our dataset, a collection of profiles presenting the list of identifying
attributes of this officer as they appear in each dataset provided by the CPD.
Since these attributes can change over time for legitimate reasons, we believe
that the collection of profiles as a whole is the most faithful and complete
representation of each officer. Working with such collections of profiles can
however be counter-intuitive and inconvenient, since some applications might
for example require to display the name of an officer, without having to choose
from possibly two or more names in cases this officer changed name over the
course of their career in the CPD. For such use cases, we consolidated the
different profiles of each officer into a single, canonical profile as follows.
For each officer's identifying attribute, we choose the \emph{most recent
nonempty} value it takes among all profiles of this officer, where \emph{most
recent} is defined using the release date of each dataset by the CPD. In this way,
if an attribute is empty in some profiles but present in others, a nonempty value
will be selected. Choosing the most recent value is justified since (1) it is
more likely to still be current (2) it is more likely to contain the latest
corrections made by the CPD to their database. The consolidated profiles
for each unique officer can be found in the file \texttt{data/roster.csv}.

\paragraph{Cleaning unit assignments.}
As already alluded to in \cref{sec:raw}, the unit assignment data revealed that
around 6\% of the records have an end date which chronologically precedes the
start date. A closer inspection of these faulty records revealed a systematic
pattern: whenever such a record appears, it is possible to find among the other
assignments of the same officer another record whose start date is exactly one
day after the end date of the faulty record. For example, displaying each
record as a triplet (\texttt{unit\_number}, \texttt{start\_date},
\texttt{end\_date}), we might find for a given officer:
\begin{quote}
\begin{verbatim}
  1 1967-12-18 1972-05-06
 22 1967-12-18 1967-12-17
\end{verbatim}
\end{quote}
where the end date for the faulty assignment to unit $22$ is one day before the
start date of the assignment to unit $1$. This led us to formulate the
following hypothesis: \emph{the faulty end dates where not manually entered but
were instead automatically generated by the data infrastructure of the CPD}.
More specifically, we believe the end dates were added by a computer code which
processed all unit assignments in order, and set as the end date of each
assignment, the day immediately preceding the start date of the following
assignment. The faulty records then arose from the fact that they were wrongly
positioned in the order considered by the computer code. The reason for the
wrong positioning of these records is that they are for the most part,
erroneous, inactive records which, we believe, should have been removed from
the dataset.

A strong supporting piece of evidence is that for 92\% of these faulty records,
we can find another record for the same officer with the same start date and
different unit number, as in the example above. An attempt at reconstructing the
true story behind this example is as follows. On 1967-12-18, someone
wrongly entered an assignment to unit $22$ for this officer, the mistake was
immediately noticed and a new, correct assignment to unit $1$ was added to the
database on the same day. When end dates were later added by the above
mentioned piece of computer code, the assignment to unit $1$ was processed
last, hence adding 1967-12-17 (the day preceding the assignment to unit $1$) as
the end date of the assignment to unit $22$. The correct solution in this case
is to simply remove the assignment to unit 22 from the dataset.

A detailed description of how the remaining 8\% of the faulty records (486
records out of the 116\,027 records in the original data) are processed can be
found in the script \texttt{src/clean\_assignments.py}.


\section{Additional summary figures}\label{sec:additional_figs}

In this section we provide additional exploratory visualizations 
of the data. \cref{fig:trrs_times} provides temporal information
about TRRs, \cref{fig:position} shows the fraction of officers 
in different positions by demographic, \cref{fig:salary_gender_race}
shows officer salary by demographic, 
and \cref{fig:complaints_times} provides additional temporal information about complaints.
\cref{fig:degree_distribution,fig:intrashooting_time} display properties 
of the social network described in \cref{sec:discussion}.

\begin{figure}[ht!] 
	\includegraphics[width=\textwidth]{figs/trrs_times} 
	\caption{TRRs filed versus time at various scales (hour, weekday, month, year).} \label{fig:trrs_times}
\end{figure}
\hfill
\begin{figure}[ht!]
	\includegraphics[width=\textwidth, clip, trim= 0 0 460 0]{figs/complaints_times} 
\caption{Complaints versus time at various scales (weekday, month)}\label{fig:complaints_times}
\end{figure}

\begin{figure}[ht!] 
	\begin{center}
\begin{subfigure}{0.47\textwidth}
\includegraphics[width=\textwidth]{figs/position_race} 
\end{subfigure}
\hfill
\begin{subfigure}{0.47\textwidth}
\includegraphics[width=\textwidth]{figs/position_gender} 
\end{subfigure}
	\end{center}
	\caption{Fraction of officers in pools of ranks (Officer, Detective, Sergeant, Cap\-tain/Com\-mander/Lieutenant, and (First/Deputy) Chief/Superintendent)
 by race (left) and gender (right).} \label{fig:position}
\end{figure}

\begin{figure}[ht!] 
	\begin{center}
	\includegraphics[width=0.8\textwidth]{figs/salary_by_race_gender} 
	\end{center}
	\caption{Salary (box indicates quartiles, whiskers indicate 5th/95th percentile, and scatter points are outliers) by race and gender.} \label{fig:salary_gender_race}
\end{figure}


\begin{figure}[ht!] 
	\includegraphics[width=\textwidth]{figs/degree_distribution} 
	\caption{Degree distribution for the complaints network.}
\label{fig:degree_distribution}
\end{figure}

\begin{figure}[ht!] 
	\includegraphics[width=\textwidth]{figs/intrashooting_times} 
	\caption{Intrashooting times between pairs of shootings.}
\label{fig:intrashooting_time}
\end{figure}






\end{document}
