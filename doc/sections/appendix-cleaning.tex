\section{Data Cleaning and Linking: Additional Details}\label{sec:app-cleaning}

\subsection{Initial Cleaning}

The files initially released by the CPD are for
the most part Excel spreadsheets, with inconsistent formatting and which can
thus be difficult to process programmatically. As an example, the reader is
invited to open the file
\texttt{p046957\_-\_report\_1.1\_-\_all\_complaints\_in\_time\_frame.xls}
available in the folder \texttt{raw/P0-46957/}. As can be seen, each record in
this file is spread over two rows of the spreadsheet, with the field names
repeated at the beginning of the second row for each record.

Consequently, the goal of the cleaning step is to produce uniformly formatted
CSV files, with the minimum requirement that each record be presented on
a single line after this step. The code is contained in the files
\texttt{datasets.py}, \texttt{parse.py}, \texttt{parse\_p046957.py}, and \texttt{parse\_p061715.py} in the
\texttt{src/} folder.  The step can be applied by running \texttt{make prepare}
in the root of the repository, which creates the folder \texttt{tidy/}
containing the cleaned CSV files.

The decisions made at this stage are straightforward and involve no subjective
judgment. They consist of:
\begin{itemize}
	\item Unifying attribute names across datasets, so that the same type of
		data is always identified in the same way (for example,
		\texttt{Appointment Date}, \texttt{Appt Date},
		\texttt{appointment\_date} are all mapped to
		\texttt{appointment\_date}).
	\item Unifying attribute values across datasets. For example, the gender of
		an officer is indicated as a single letter \texttt{M}/\texttt{F} in
		some datasets, and as \texttt{Male}/\texttt{Female} in others. A similar issue
		arises with the race of officers, sometimes given as a three-letter
		code, and sometimes described in full. In such cases we map all
		possible forms of an attribute value to a canonical one. \footnote{Probably for space, can merge these two bullet points as ``Unifying attribute names and values across datasets''.}
	\item Parsing values into the correct data type or format. For example,
		dates and times are formatted differently depending on the dataset, and
		we map everything to the ISO\,8601 format. Integers are also parsed so
		as to remove the various paddings present in the original data.
	\item Concatenating all the files containing a given type of record in each
		data release. Indeed, the CPD's responses to some FOIA requests split
		the records chronologically over multiple Excel spreadsheets (see for
		example \texttt{raw/P0-46957}) or multiple tables within spreadsheets
                (see for example \texttt{raw/salary}). Having a single file for each record
		type instead simplifies later processing steps.
\end{itemize}

Although this initial cleaning is only the first step in the process producing
our final dataset, we structured our processing code in such a way that it is
easy to stop the process at this step and keep the intermediate output in the
\texttt{tidy/} folder. This is because the next steps required making more
subjective judgment calls, e.g., to decide how to clean erroneous records and resolve
ambiguities arising from the merging and linking of datasets. Consequently, it
is possible that some applications will require performing these next steps
differently. In such cases, the files contained in the \texttt{tidy/} folder
should provide a safe intermediate point at which to branch off from our
processing pipeline.

\subsection{Linking and Merging Datasets}\label{sec:linking}
\label{sec:linking}

\paragraph{Overall description.}

As discussed in \cref{sec:raw}, the main challenge in processing the raw data
is that officers are not uniquely identified across datasets. In other words,
there is no foolproof way to know if two different records from two different
datasets correspond to the same individual. For example, a na\"ive matching
procedure identifying two officers as being the same individual whenever they
have the same name is inadequate: given the size of the CPD roster (over 35\,000
active or retired officers), it is guaranteed to contain many homonymous
individuals. Furthermore, even though the original datasets contain several
identifying attributes (name, birth year, appointment date, race, gender),
those can change over item as mentioned in \cref{sec:raw}.

At a high level, our procedure \emph{grows} a population of uniquely identified
officers by sequentially examining each FOIA release. Each unique officer in
this population is assigned a unique identification (UID), which is a random
hexadecimal string (e.g.\ \texttt{9bc51eef-c37b-4eff-a14d-7e69f56b3d1e}), and
to each UID is associated a list of \emph{officer profiles}. There is one
\emph{officer profile} for each unique officer and each FOIA release in which
this officer appeared, listing the identifying attributes of this officer as
they appear in this given data release. In detail, for each FOIA release:
\begin{enumerate}
	\item We build a list of all the \emph{officer profiles} appearing in this
		release.
	\item For each officer profile, we attempt to \emph{match} it against the
		profiles of the population of unique officers constructed so far from
		previously examined FOIA releases.
		\begin{itemize}
			\item If the match is successful, we have identified a unique officer in
				the population whose profiles unambiguously match with the
				current profile. In this case, we simply attach the current
				profile to this officer and UID, and
				the population does not grow.
			\item If the match is unsuccessful, we add a new officer with a new UID to the population of unique
				officers and attach the current profile to this new officer.
		\end{itemize}
\end{enumerate}

After all FOIA releases have been processed, the population contains the set of
all unique officers appearing in the original data. Each officer is represented
by UID and a collection of profiles, representing the various ways in which
this unique officer appears across different datasets. These profiles can be
found in the file \texttt{final/officer\_profiles.csv}.
At this point the \texttt{final/} folder also contains
one file for each type of record (complaints, tactical response reports, etc).
Within these files, officers are identified solely by their UID and other attributes
are removed; this avoids duplication of
information since these attributes are redundant with those found in
\texttt{final/officer\_profiles.csv}.

Note that this procedure depends on the order in which the FOIA releases are processed,
since each release will be matched against the profiles from all
previously considered releases. We chose to first process the two roster datasets
(\texttt{P0-58155} and \texttt{P4-41436}) since they are most similar and
supposed to contain one record for each officer in the CPD. Next, we process
the two unit assignments datasets (\texttt{16-1105} and \texttt{P0-52262})
since they are also supposed to cover the entire CPD. Finally, we process the
complaints, tactical response reports, awards, and salary data, in this
order.

\paragraph{Iterative pairwise matching.}
It remains to describe the \emph{match} operation which was left unspecified in
step 2.\ of the procedure above. This operation
matches a list of profiles found in the FOIA release being currently processed
against an already existing list of profiles and associated UIDs. We need to strike a balance between:
\begin{itemize}
	\item Being loose enough to avoid type II error (false negatives). If the
		same officer appears with slightly different attributes in two
		different profiles, we do not want the matching method to believe these
		profiles are attached to different officers.
	\item Being strict enough to avoid type I error (false positives). We do
		not want to attach a profile to an officer currently in the population
		if it in fact corresponds to a new officer.
\end{itemize}

We developed an \emph{iterative pairwise matching procedure}\footnote{This
procedure was inspired by a similar one developed by the Invisible Institute.},
that makes iterative passes over the list $L_1$ of profiles to be matched
against the list $L_2$ of already existing profiles. Recall that in our case,
the profiles in $L_2$ are associated with officers in the growing population of
unique officers, and hence they are each associated with a UID. During each
pass:
\begin{enumerate}
	\item A subset $S$ of the profile attributes is selected as the matching
		criterion.
	\item For each profile $p$ in $L_1$:
		\begin{enumerate}
			\item Construct the list $\ell_p$ of profiles in $L_2$ for which
				the attributes in $S$ match with $p$ exactly.
			\item If all the profiles in $\ell_p$ are associated with the same
				UID $u$, this is an unambiguous match and we can safely assign
				the UID $u$ to the profile $p$. We remove $p$ from $L_1$ and
				all the profiles associated with $u$ in $L_2$.
			\item Otherwise, the match is ambiguous and we keep $p$ in
				$L_1$.
		\end{enumerate}
\end{enumerate}
Observe that both $L_1$ and $L_2$ decrease in size as matches are found, so
each pass iterates over a smaller set of profiles than the previous one. When
all passes are done, the remaining profiles in $L_1$ are considered new
unique officers and assigned freshly generated UIDs. Furthermore, by
building at each pass a hash table mapping a subset of attributes to the
list of profiles in $L_2$ sharing these attributes, the construction of
$\ell_p$ in 2.(a) can be performed in constant time. The overall running time of the
  iterative matching procedure is $O\big(P(|L_1|+|L_2|)\big)$ where $P$ is the
  number of passes.\footnote{\textcolor{red}{If we are talking about running time, I think we could also mention that - in practice - this procedure runs in less than a minute (or actually, check how long it takes) on a standard machine?}}

Finally, to fully specify the matching procedure we need to describe how to
choose which subset $S$ of identifying attributes is chosen as the matching
criterion at step 1.\ during each pass. For this, we go from the stricter to
the looser criterion: in the first pass, $S$ contains \emph{all} the attributes
which are present in both $L_1$ and $L_2$, and then attributes are removed from
$S$ one by one in subsequent passes. For example, one can remove the \emph{last
name} attribute in the second pass to match a pair of profiles whose last names
are different but match on all the remaining attributes, thus identifying an
officer whose last name changed between two FOIA releases. The advantage of
going from stricter to looser is twofold:
\begin{itemize}
	\item Starting from the strictest set of attributes identifies the
		\emph{least ambiguous matches}, for which profiles match exactly on
		a large set of attributes and can confidently be considered to describe
		the same individual. This constitutes the vast majority of cases;
		more than 95\% of profiles are usually matched during the first pass.
		Since the next passes iterate only on the remaining profiles, this
		removes the majority of ambiguities that could arise as the
		matching criterion is relaxed.
	\item Since the vast majority of profiles are matched in the first pass,
		it becomes feasible in subsequent passes to visually inspect all
		the matched profiles and assess whether the chosen set of attributes was
		too strict or too lose.
\end{itemize}

We note that there is some amount of subjective judgement involved;
 visual inspection is employed in the second and subsequent passes to decide which sets
attributes are stringent enough as to avoid accidentally matching two different persons. 
This is also how we decide that sufficiently many passes have been
performed and that further relaxing the matching criterion would introduce too
many type I errors. In general, we erred on the side of favoring
type II errors over type I errors when uncertain.

%\textbf{TODO:} explain the few subtleties where we don't use equality to match
%attributes (for example when matching age and birthyear, where the age only
%lets us identify the birthyear with an accuracy of 1 year). Or with stars,
%where we test whether a star number is contained in the subset of known stars
%for this officer.

%\textbf{TODO:} table of officer attributes in each dataset


\subsection{Final cleaning and output}

\paragraph{Roster consolidation.}
The procedure described in \cref{sec:linking} produces, for each unique officer
in our dataset, a collection of profiles presenting the list of identifying
attributes of this officer as they appear in each dataset provided by the CPD.
Since these attributes can change over time for legitimate reasons, we believe
that the collection of profiles as a whole is the most faithful and complete
representation of each officer. Working with such collections of profiles can
however be counter-intuitive and inconvenient, since some applications might
for example require to display the name of an officer, without having to choose
from possibly two or more names in cases this officer changed name over the
course of their career in the CPD. For such use cases, we consolidated the
different profiles of each officer into a single, canonical profile as follows.
For each officer's identifying attribute, we choose the \emph{most recent
nonempty} value it takes among all profiles of this officer, where \emph{most
recent} is defined using the release date of each dataset by the CPD. In this way,
if an attribute is empty in some profiles but present in others, a nonempty value
will be selected. Choosing the most recent value is justified since (1) it is
more likely to still be current (2) it is more likely to contain the latest
corrections made by the CPD to their database.\footnote{This assumes that it
is, on average, less likely for errors to be introduced in the CPD database,
than for records to be fixed. This assumption is highly debatable and the
authors' posterior belief after working on the CPD data for more than a year
assigns probability at most $0.6$ to its veracity.} The consolidated profiles
for each unique officer can be found in the file \texttt{data/roster.csv}.

\paragraph{Cleaning unit assignments.}
As already alluded to in \cref{sec:raw}, the unit assignment data revealed that
around 6\% of the records have an end date which chronologically precedes the
start date. A closer inspection of these faulty records revealed a systematic
pattern: whenever such a record appears, it is possible to find among the other
assignments of the same officer another record whose start date is exactly one
day after the end date of the faulty record. For example, displaying each
record as a triplet (\texttt{unit\_number}, \texttt{start\_date},
\texttt{end\_date}), we might find for a given officer:
\begin{quote}
\begin{verbatim}
  1 1967-12-18 1972-05-06
 22 1967-12-18 1967-12-17
\end{verbatim}
\end{quote}
where the end date for the faulty assignment to unit $22$ is one day before the
start date of the assignment to unit $1$. This led us to formulate the
following hypothesis: \emph{the faulty end dates where not manually entered but
were instead automatically generated by the data infrastructure of the CPD}.
More specifically, we believe the end dates were added by a computer code which
processed all unit assignments in order, and set as the end date of each
assignment, the day immediately preceding the start date of the following
assignment. The faulty records then arose from the fact that they were wrongly
positioned in the order considered by the computer code. The reason for the
wrong positioning of these records is that they are for the most part,
erroneous, inactive records which, we believe, should have been removed from
the dataset.

A strong supporting piece of evidence is that for 92\% of these faulty records,
we can find another record for the same officer with the same start date and
different unit number, as in the example above. An attempt at reconstructing the
true story behind this example is as follows. On 1967-12-18, someone
wrongly entered an assignment to unit $22$ for this officer, the mistake was
immediately noticed and a new, correct assignment to unit $1$ was added to the
database on the same day. When end dates were later added by the above
mentioned piece of computer code, the assignment to unit $1$ was processed
last, hence adding 1967-12-17 (the day preceding the assignment to unit $1$) as
the end date of the assignment to unit $22$. The correct solution in this case
is to simply remove the assignment to unit 22 from the dataset.

A detailed description of how the remaining 8\% of the faulty records (486
records out of the 116\,027 records in the original data) are processed can be
found in the script \texttt{src/clean\_assignments.py}.

\paragraph{Final output.}
