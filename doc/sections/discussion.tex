% !TEX root = ../main.tex
\section{Discussion} \label{sec:discussion}

\subsection{Intended Uses}
Uses for this dataset are varied and rich. For example, researchers could use
the data in a wide range of predictive tasks, such as predicting officer
misconduct, resignation, and shooting as a function of their underlying
demographic data or complaints filed against them. Past work has engaged in
this type of analysis on both the raw data underlying the present work, as well
as confidential internal police department data \cite{Helsby18,Rozema19}.  Our
dataset can also be used to study social networks (both in the context of
policing and more generally). In particular, we can use the complaint data in
\texttt{complaints\_officers.csv} to construct an undirected graph on the set
of police officers with an edge between each pair of officers listed together
on the same complaint. Moreover, the complaints can be linked to the
\texttt{tactical\_response\_reports.csv} file to focus on the subgraph of
officers who filed a TRR. We report summary statistics for the corresponding
graphs in \Cref{tab:stats_graphs}, with additional related visualizations in
\cref{sec:additional_figs}.  These networks are of interest in and of
themselves, but can also be used to investigate the dynamic patterns of officer
wrongdoing along such police networks \cite{Roithmayr16}. Existing research has
used the complaint data to identify such patterns and to investigate whether
pairs of officers connected on a network are more likely to have been accused
of misconduct \cite{Ouellet19}.  Finally, this dataset could be used to track
the effects of new disciplinary practices, new training techniques, and new
oversight on complaints and use of force. A working paper explores, for
example, whether civilians filed fewer complaints about officers' force in the
wake of the Department of Justice investigation of the Chicago Police
Department \cite{Travers20}. 


\begin{table}[t!]
	\begin{center}
\caption{Summary statistics for the complaints network graph, and the subgraph
of officers in TRRs. The network is constructed from all complaints and TRRs filed between 
2004-01-01 and 2015-12-01. $N$ and $E$ denote the number of nodes and edges respectively. $N_\ell$ and $N_i$ denote respectively the number of nodes in the largest connected component and the number of isolated nodes.}\label{tab:network}
\vspace{0.5em}
\begin{tabular}{c|c|c|c|c|c|c|c|}
\cline{2-8}
                                                & $N$ & $E$ & \textit{Avg. degree} & \textit{Triangles} & \textit{Max clique} & $N_\ell$ & $N_i$ \\ \hline
\multicolumn{1}{|c|}{\textit{\textbf{All}}}     & $14{,}372$      & $106{,}701$     & $14.85$              & $361{,}878$        & $64$                & $13{,}950$   & $0$                   \\ \hline
\multicolumn{1}{|c|}{\textit{\textbf{In TRRs}}} & $4{,}105$       & $22{,}064$      & $10.75$              & $44{,}786$         & $28$                & $3{,}822$    & $225$                 \\ \hline
\end{tabular} \label{tab:stats_graphs}
	\end{center}
\end{table}

\subsection{Ethical Considerations}
Recent work on algorithmic fairness focuses on the potential for racially
biased data to produce racially biased results
\cite{veale2018fairness,sloane2019ai,d2020fairness}.  This research suggests
that race shapes data collection in criminal justice, in at least two ways that
are likely to affect data collection on black officers. First, beginning back
in the 1960s, black officers are more likely to be assigned to black
neighborhoods and/or to neighborhoods where police interaction is more
pervasive \cite{Kuykendall80}. Given an increased frequency of interaction,
officers assigned to these neighborhoods may be statistically more likely to be
the subject of complaints \cite{Kane06}.  Second, owing to cognitive bias, complainants 
may be more likely to file a complaint against black officers, either alone or in pairs.
\cref{fig:complaints} provides initial quantitative evidence towards that point. This racial
asymmetry in the collection of complaint data may well produce, for example,
racially biased predictions of police misconduct. Relatedly, because our
dataset may be used to explore predictive policing of the police, black
officers may be unfairly and disproportionately identified to be at higher risk
of misconduct \cite{veale2018fairness,sloane2019ai,d2020fairness,Wood19}. 

\subsection{Limitations and Future Work}

\looseness=-1
Using civilian and administrator complaint data to study actual (not merely
perceived) police misconduct inevitably faces significant questions about
validity. At least one study has found that because of flaws in police record
keeping and categorization, the practice of using complaints to measure police
behavior is unreliable \cite{Hickman16}. Even so, other research finds a strong
correlation between civilian-filed complaints against officers and internal
complaints against the same officers filed by other officers or supervisors
\cite{Lersch00}. Whatever the truth of the matter, the dataset could be
strengthened by adding more objective measures of misconduct, for example, data
on individual officer misconduct from oversight agencies.

Use of this dataset for network research faces a particular set of limitations.
To wit, researchers who use complaint data to generate social networks must
acknowledge that the co-listing of two officers on a complaint is an
 incomplete proxy for a professional network relationship or
exposure to another officer's misconduct. Data on partner assignment and
dispatches would more accurately reflect officer relationships and their
exposure to misconduct. 

In general, future work should focus on integrating into this dataset as many
objective sources of data on individual officers as possible. Objective data
could include information about adverse incident histories, officer discipline
histories, counseling interventions, domestic violence incidents, weapons
violations, sustained complaints, and lawsuit settlements. Additional information
about officer activities could include partner assignments, dispatch
information, arrest and stop information, unit leadership, and unit disciplinary history. 
