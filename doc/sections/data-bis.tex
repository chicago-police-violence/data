\paragraph{Roster consolidation.}
The procedure described in \cref{sec:linking} produces, for each unique officer
in our dataset, a collection of profiles presenting the list of identifying
attributes of this officer as they appear in each dataset provided by the CPD.
Since, these attributes can change over time for legitimate reasons, we believe
that the collection of profiles as a whole is the most faithful and complete
representation of each officer. Working with such collections of profiles can
however be counter-intuitive and inconvenient, since some applications might
for example require to display the name of an officer, without having to choose
from possibly two or more names in cases this officer changed name over the
course of their career in the CPD. For such use cases, we consolidated the
different profiles of each officer into a single, canonical profile as follows.
For each officer's identifying attribute, we choose the \emph{most recent
nonempty} value it takes among all profiles of this officer, where \emph{most
recent} is defined using the release date of each dataset by the CPD. This way,
if an attribute is empty in some profiles but not in others, a nonempty value
will be selected. Choosing the most recent value is justified since (1) it is
more likely to still be current (2) it is more likely to contain the latest
corrections made by the CPD to their database.\footnote{This assumes that it
is, on average, less likely for errors to be introduced in the CPD database,
than for records to be fixed. This assumption is highly debatable and the
authors' posterior belief after working on the CPD data for more than a year
assigns probability at most $0.6$ to its veracity.} The consolidated profiles
for each unique officer can be found in the file \texttt{data/roster.csv}.

\paragraph{Cleaning unit assignments.} As already alluded to in \cref{sec:raw},
an inspection of the unit assignment data revealed that a significant fraction
of the records records 
