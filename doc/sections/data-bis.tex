\paragraph{Roster consolidation.}
The procedure described in \cref{sec:linking} produces, for each unique officer
in our dataset, a collection of profiles presenting the list of identifying
attributes of this officer as they appear in each dataset provided by the CPD.
Since, these attributes can change over time for legitimate reasons, we believe
that the collection of profiles as a whole is the most faithful and complete
representation of each officer. Working with such collections of profiles can
however be counter-intuitive and inconvenient, since some applications might
for example require to display the name of an officer, without having to choose
from possibly two or more names in cases this officer changed name over the
course of their career in the CPD. For such use cases, we consolidated the
different profiles of each officer into a single, canonical profile as follows.
For each officer's identifying attribute, we choose the \emph{most recent
nonempty} value it takes among all profiles of this officer, where \emph{most
recent} is defined using the release date of each dataset by the CPD. This way,
if an attribute is empty in some profiles but not in others, a nonempty value
will be selected. Choosing the most recent value is justified since (1) it is
more likely to still be current (2) it is more likely to contain the latest
corrections made by the CPD to their database.\footnote{This assumes that it
is, on average, less likely for errors to be introduced in the CPD database,
than for records to be fixed. This assumption is highly debatable and the
authors' posterior belief after working on the CPD data for more than a year
assigns probability at most $0.6$ to its veracity.} The consolidated profiles
for each unique officer can be found in the file \texttt{data/roster.csv}.

\paragraph{Cleaning unit assignments.}
As already alluded to in \cref{sec:raw}, the unit assignment data revealed that
around 6\% of the records have an end date which chronologically precedes the
start date. A closer inspection of these faulty records revealed a systematic
pattern: whenever such a record appears, it is possible to find among the other
assignments of the same officer another record whose start date is exactly one
day after the end date of the faulty record. For example, displaying each
record as a triplet (\texttt{unit\_number}, \texttt{start\_date},
\texttt{end\_date}), we might find for a given officer:
\begin{quote}
\begin{verbatim}
  1 1967-12-18 1972-05-06
 22 1967-12-18 1967-12-17
\end{verbatim}
\end{quote}
where the end date for the faulty assignment to unit $22$ is one day before the
start date of the assignment to unit $1$. This led us to formulate the
following hypothesis: \emph{the faulty end dates where not manually entered but
were instead automatically generated by the data infrastructure of the CPD}.
More specifically, we believe the end dates were added by a computer code which
processed all unit assignments in order, and set as the end date of each
assignment, the day immediately preceding the start date of the following
assignment. The faulty records then arose from the fact that they were wrongly
positioned in the order considered by the computer code. The reason for the
wrong positioning of these records is that they are for the most part,
erroneous, inactive records which, we believe, should have been removed from
the dataset.

A strong supporting piece of evidence is that for 92\% of these faulty records,
we can find another record for the same officer with the same start date and
different unit number, as in the example above. An attempt at reconstructing the
true story behind this example is as follows. On 1967-12-18, someone
wrongly entered an assignment to unit $22$ for this officer, the mistake was
immediately noticed and a new, correct assignment to unit $1$ was added to the
database on the same day. When end dates were later added by the above
mentioned piece of computer code, the assignment to unit $1$ was processed
last, hence adding 1967-12-17 (the day preceding the assignment to unit $1$) as
the end date of the assignment to unit $22$. The correct solution in this case
is to simply remove the assignment to unit 22 from the dataset.

A detailed description of how the remaining 8\% of the faulty records (486
records out of the 116\,027 records in the original data) are processed can be
found in the script \texttt{src/clean\_assignments.py}.

