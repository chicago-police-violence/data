\paragraph{Roster consolidation.}
The procedure described in \cref{sec:linking} produces, for each unique officer
in our dataset, a collection of profiles presenting the list of identifying
attributes of this officer as they appear in each dataset provided by the CPD.
Since, these attributes can change over time for legitimate reasons, we believe
that the collection of profiles as a whole is the most faithful and complete
representation of each officer.

Working with such collections of profiles can however be counter-intuitive and
inconvenient, since some applications might for example require to display the
name of an officer, without having to choose from possibly two or more names in
cases this officer changed name over the course of their career in the CPD. For
such use cases, we consolidated the different profiles of each officer into
a single, canonical profile as follows. For each officer's identifying
attribute, we choose the \emph{most recent nonempty} value it takes among all
profiles of this officer, where \emph{most recent} is defined using the release
date of each dataset by the CPD. This way, if an attribute is empty in some
profiles but not in others, a nonempty value will be selected. Choosing the
most recent value is justified since (1) it is more likely to still be current
(2) it is more likely to contain the latest corrections made by the CPD to
their database.\footnote{This assumes that on average, it is less likely for
errors to be introduced in the CPD database, than for records to be fixed. This
assumption is highly debatable and the authors' posterior belief after working
on the CPD data for more than a year assigns at most $0.6$ probability to its
veracity.}



This consolidated profile was obtained as follows:
for 



Thus 

A byproduct of matching officers across datasets is that for each officer, we
now have as many “profiles” as the number of datasets in which they appear,
where by \emph{profile} we mean a collection of attributes. Note that each
profile can contain a different subset of attributes (since not all attributes
are present in each dataset) and that a given attribute might take a different
value in different profiles of the same officer (since the iterative matching
procedure is not restricted to performing strict matching).

We are thus faced with the task of “consolidating” the different profiles of
a given officer into a single profile. Of course, if an attribute is present in
a single profile, and absent from the others, this is the value we keep in the
consolidated profile. But if an attribute appears with different values across
different profiles, we choose the value coming from the profile corresponding
to the \emph{most recent data release}.

\paragraph{Cleaning unit assignments.}
