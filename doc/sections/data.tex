\section{The CPD Data} \label{sec:data}

The original raw data released by the CPD as well as the code to generate the
cleaned data, and the source code for this document are available in a GitHub
repository at \url{https://github.com/chicago-police-violence/data}. This
repository will serve as a long-term home for this data, its current and future
releases, as well as discussions regarding improvements and extensions of the
data processing code.



\subsection{The raw data: origin, description, and challenges}

\paragraph{Origin.}
The first raw data files in this repository were obtained by J.~Kalven, an 
independent journalist, who filed Illinois Freedom of Information Act requests with 
the Chicago Police Department regarding complaints filed against officers. 
In Kalven v.~City of Chicago \cite{kalven2014}, an Illinois appellate court issued
a general ruling that documents bearing on allegations of
police abuse are public information. Following 
the decision, the non-profit
Invisible Institute began to collaborate with Kalven 
and the University of Chicago's Mandel Legal Aid
Clinic to follow up on earlier FOIA requests and to file new ones. The data
disclosed in response to these earlier and now ongoing FOIA requests were made available
online as part of the Citizens Police Data Project \cite{cpdp}.
These data form the basis of the cleaned and linked data set provided by the present work.

\begin{table}[h]
	\begin{center}
\caption{Summary of the FOIA requests contained in our repository (blanks are missing entries).}
\label{table:summary}
\begin{tabular}{@{}llllr@{}}
	\toprule
	request \#&received&requested&description& \# records\\
\midrule
	\texttt{P0-58155}&2017-04-17& &Officer roster& 32\,446\\
	\texttt{P4-41436}&2018-03-21& &Officer roster& 14\,634\\
	\texttt{16-1105}&2016-03-11&2016-02-10&Unit assignments&114\,630\\
	\texttt{P0-52262}&2016-12-04&2016-09-19&Unit assignments&115\,987\\
	\texttt{P0-46957}&2016-06-29&2016-04-22&Complaints (CPD)&109\,339\\
	\texttt{18-060-425}&2018-08-28&2018-08-20&Complaints (COPA)&182\,337\\
	\texttt{P0-46360}& & &Tactical Response Reports&67\,019\\
	\texttt{P0-46987}&2016-05-13&2016-04-25&Unit names&237\\
	\texttt{P0-61715}& &2017-07-26&Awards&699\,912\\
	\texttt{P5-06887}&2019-10-11&2019-07-19&Awards&60\,556\\
					 &2017-09-27&2017-09-13&Salary&212\,508\\
\bottomrule
\end{tabular}
\end{center}
\end{table}

\paragraph{Description.}
The raw data files are contained in the \url{raw/} folder of the
repository. Each subfolder corresponds to a FOIA request, which is
generally identified by a request number. \cref{table:summary} gives
an overview of all the requests that we include in
our repository; this meta-information is also included in the \url{raw/datasets.csv}
file in the repository. The subfolders contain the data provided by the city
in response to their corresponding FOIA requests, which typically comprises multiple
Excel spreadsheets. In addition, when available, the subfolders contain formal correspondence 
regarding the request, which often provides useful contextual information in understanding the data.  

In particular, the raw data files contained in the repository
provide the following information: {\color{red} todo: year ranges for each?}
\begin{description}
	\item[Officer roster:] a list of all officers (past and present) employed
		by the CPD along with attributes such as year of birth, age, race,
		gender, appointment date, resignation date, etc.
	\item[Unit assignments:] the CPD is organized into over 200 units.
		Each officer can be assigned to one or multiple units and these
		assignments can change over time. The unit assignment datasets contain
		one record for each officer and each unit they were assigned to, with
		the start date and end date of this assignment.
	\item[Complaints:] formal complaints against police officers, filed both by
		citizens and internally within the department. Complaints are
		identified by a complaint number. There is one record for each
		complaint and each officer listed on the complaint, indicating the
		allegation made against them, result of the investigation of the
		allegation (with possible sanction), etc.
	\item[Tactical Response Reports:] these are forms that officers are
		required to file after each incident for which the officer's response
		involved use of force. There is one record for each incident and each
		officer involved in the incident. Each record contains details about
		the incident (such as time and location), the officer involved and the
		subject of the use of force. In case one or multiple weapons were used,
		detailed information about each use is also provided including, for
		firearms, the number of discharges and the object struck at each
		discharge.
	\item[Unit names:] the (human-readable) name of each (past and present)
		unit in the CPD. These names provide information about the function of
		each unit and also appear occasionally where unit numbers are listed in
		the other data files.
	\item[Awards:] a list of all awards requested for officers in the CPD,
		including award tracking number, reference number, award type, request
		date, requester name, etc.
	\item[Salary:] a list of officers from 2002--2017 including their salary,
		position, and pay grade.
\end{description}

\paragraph{Challenges.}
Despite the richness of the information contained in these FOIA data releases,
there is a major obstacle to using them for investigating the activities of the
CPD. Police officers are not uniquely identified across datasets, so there is
a priori no reliable way to know whether, for example, an officer listed on
a Complaint is the same individual as an officer with similar attributes listed
on a Tactical Response Report. Although the CPD almost certainly assigns
a unique identifier to each of its officers internally, these identifiers were
never publicly released in the responses to FOIA requests. Linking officers
across datasets must thus be performed by using the restricted set of
attributes present in the dataset, and comes with the following challenges:
\begin{itemize}
	\item \emph{Time-varying attributes:} many attributes change over the
		course of an officer's career in the CPD, such as their unit
		assignments, rank or badge number (star). Perhaps surprisingly some
		attributes which would usually be considered stable and useful
		identifiers also change over time. These include surnames (e.g.\ when
		marrying) or appointment dates (e.g.\ when a database entry is
		corrected internally).
	\item \emph{Multiple internal sources:} the salary data comes from
		a different database. In particular, the officers' names were entered
		separately, which makes linking this data to the other datasets
		particularly challenging.
	\item \emph{Inconsistent entries:} various choices were made by the CPD to
		decide which officers to include in each dataset.  There are, for
		example, officers missing from the roster or unit assignment data, but
		present in the salary data. Furthermore, the same attribute can appear
		under different names in different datasets and sometimes have
		ambiguous meanings: for example, the salary data contains two different
		attributes for the appointment date.
	\item \emph{Duplicate entries:} probably due to internal errors in the CPD
		data infrastructure, some officers are sometimes duplicated in the
		roster and unit assignment data: they appear twice in the same dataset,
		as two different individuals but with the exact same attributes. One of
		the two “copies” of each duplicate officer is inactive and never
		appears in the rest of the data, but introduces ambiguities to uniquely
		identify officers across datasets.
	\item \emph{Systematic errors:} an unusual difficulty arose from the unit
		assignment data, in which a significant fraction of the assignments
		have an end date chronologically preceding the start date. This appears
		to be a systematic error, either in the data entry process or in the
		code which produced the data. A close inspection of the pattern of
		errors revealed that the faulty records cannot be fixed by simply
		swapping the start date with the end date.
\end{itemize}

A major contribution of the present work is to address all the above challenges
by carefully cleaning and linking the original datasets. The output of this
process is a collection of comma-separated values (CSV) files corresponding to
the different entities described in the \emph{Description} paragraph above.
Importantly, each officer is uniquely identified by a hexadecimal string across
output files.

The data processing code is written in Python, and the documentation is written
in \LaTeX.  \texttt{Make} is used to coordinate the various processing steps
and easily reproduce them.  Refer to \url{README.md} in the repository for
details regarding system requirements and running the data processing code. The
remainder of this section describes the processing steps in detail.

\subsection{Initial Cleaning}

The files initially released by the CPD are for
the most part Excel spreadsheets, with inconsistent formatting and which can
thus be difficult to process programmatically. As an example, the reader is
invited to open the file
\texttt{p046957\_-\_report\_1.1\_-\_all\_complaints\_in\_time\_frame.xls}
available in the folder \texttt{raw/P0-46957/}. As can be seen, each record in
this file is spread over two rows of the spreadsheet, with the field names
repeated at the beginning of the second row for each record.

The goal of the cleaning step is thus to produce uniformly formatted CSV files,
with the minimum requirement that each record be presented on a single line
after this step. The code is contained in the files \texttt{datasets.py},
\texttt{parse.py} and \texttt{parse\_p046957.py} in the \texttt{src/} folder.
The step can be applied by running \texttt{make prepare} in the root of the
repository, which creates the folder \texttt{tidy/} containing the cleaned CSV
files.

The decisions made at this stage are straightforward and involve no subjective
judgment. They consist of:
\begin{itemize}
	\item unifying attribute names across datasets, so that the same type of
		data is always identified in the same way (for example,
		\texttt{Appointment Date}, \texttt{Appt Date},
		\texttt{appointment\_date} are all mapped to
		\texttt{appointment\_date}).
	\item unifying attribute values across datasets. For example, the gender of
		an officer is indicated as a single letter \texttt{M}/\texttt{F} in
		some datasets or as \texttt{Male}/\texttt{Female}. A similar issue
		arises with the race of officers, sometimes given as a three-letter
		code, and sometimes described in full. In such cases we map all
		possible forms of a given attribute value to a canonical one.
	\item parsing values into the correct data type or format. For example,
		dates and times are formatted differently depending on the dataset, and
		we map everything to the ISO\,8601 format. Integers are also parsed so
		as to remove the various paddings present in the original data.
	\item concatenating all the files containing a given type of record in each
		data release. Indeed, the CPD's responses to some FOIA requests split
		the records chronologically over multiple Excel spreadsheets (see for
		example \texttt{raw/P0-46957}). Having a single file for each record
		type instead simplifies later processing steps.
\end{itemize}

Although this initial cleaning is only the first step in the process producing
our final dataset, we structured our processing code in such a way that it is
easy to stop the process at this step and keep the intermediate output in the
\texttt{tidy/} folder. This is because the next steps required making more
subjective judgment calls, to decide how to clean erroneous records and resolve
ambiguities arising from the merging and linking of datasets. Consequently, it
is possible that some applications will require performing these next steps
differently. In such cases, the files contained in the \texttt{tidy/} folder
should provide a safe intermediate point at which to branch off from our
processing pipeline.

\subsection{Linking and Merging Datasets}\label{sec:linking}

As already alluded to, the main challenge at this step is that there is no
identifier uniquely identifying officers across records. In other words, there
is no foolproof way to know if two different records correspond to the same
officer, \emph{even within the same dataset} (for example the same officer
could appear with slightly different attributes on two different complaint
records).

We thus need to design a matching method striking a balance between
\begin{itemize}
	\item being loose enough to avoid type II error (false negatives). If the
		same officer appears with slightly different attributes across two
		records, we do not want our matching method to believe it is two
		different officers.
	\item being strict enough to avoid type I error (false positives). We do
		not want to merge two different officers into a single identity.
\end{itemize}

The difficulty in achieving this balance is that perhaps surprisingly
\emph{none of the attributes of a given police officer are guaranteed to be
stable over time}. Most notably, officers' names change over time, for example
to fix data entry errors or in case of legal name changes. However, we observed
two attributes, present in almost all original datasets and which seem
remarkably stable over the time: \emph{appointment date} and \emph{birthyear}.
These two attributes thus proved very valuable to disambiguate officers with
identical names.

In order to match officers across two datasets, we developed an \emph{iterative
pairwise matching procedure} that makes iterative passes over the datasets.
During each pass, a subset of the officer attributes present in both datasets
is selected as the matching criterion, and a pair of officers (one from each
dataset) is identified as a \emph{match} if (i) their attributes from the
chosen subset match, and (ii) if they are the only two officers matching on
these attributes. Once a pair is identified as a match, it is put aside, and
the next pass is performed on the remaining unmatched officers. After all the
passes are done the leftover officers are declared as different officers. By
constructing a hash table mapping a subset of attributes to the list of
officers sharing these attributes, each pass can be performed in linear time,
so the overall running time of the procedure is $O\big(P(N_1+N_2)\big)$ where
$P$ is the number of passes and $N_1, N_2$ are the number of officers in each
dataset.

To fully specify the matching procedure we thus need to specify which subset of
attributes is chosen as the matching criterion at each pass. For this, we go
from the stricter to the looser criterion: for the first pass, we choose
a subset \emph{all} the attributes which are present in two datasets, and then
start removing attributes one by one. For example, one can remove the
\emph{last name} attribute for the second pass to match a pair of officers
whose last names are different but match on all the remaining attributes, thus
identifying an officer whose last name changed between the releases of the two
datasets. The advantage of going from stricter to looser is two fold:
\begin{itemize}
	\item starting from the strictest set of attributes identifies the
		\emph{unambiguously matching pairs}, that is, the ones which are
		a clear match and which, thankfully, constitute the vast majority of
		officers (typically around 80-90\% of officers are matched during the
		first pass \textbf{TODO check}). Since the next passes will only be
		performed on the remaining unmatched officers, this removes a lot of
		potential ambiguities which could occur once the set of attributes is
		reduced.
	\item since the vast majority of officers is matched during the first pass,
		it becomes feasible during the subsequent passes to visually inspect all
		the matched pairs and assess whether the chosen set of attributes was
		too strict or too lose (\textbf{TODO:} explain how to activate
		debugging information in the code).
\end{itemize}

We note that there is still some amount of subjective judgment involved,
following a visual inspection of the second and subsequent passes, to decide
which sets attributes are ``acceptable'' (that is, for which the probability of
two persons sharing these attributes in a population of the size of the CPD is
extremely small). This is also how we decide that sufficiently many passes have
been performed: when it would seem likely to introduce a type I error by
matching any of the remaining officers. As a general rule of thumb we erred on
the side of favoring type II errors over type I errors. That is, we only
matched officers when it would seem extremely unlikely that they correspond to
two different individuals).

\textbf{TODO:} table of officer attributes in each dataset

\textbf{TODO:} explain the few subtleties where we don't use equality to match
attributes (for example when matching age and birthyear, where the age only
lets us identify the birthyear with an accuracy of 1 year). Or with stars,
where we test whether a star number is contained in the subset of known stars
for this officer.

With this procedure at hand we can thus link officers across datasets starting
from the most similar datasets first (for which we expect to have the least
amount of ambiguity). That is we first link \texttt{P0-58155} to
\texttt{P4-41436}, then \texttt{P0-52262} to \texttt{16-1105} and then the
remaining datasets \textbf{expand}

\textbf{TODO} give in appendix table summarizing which subset of attributes are
used at each linking operation 

\subsection{Final cleaning and output}

\paragraph{Roster consolidation.}

A byproduct of matching officers across datasets is that for each officer, we
now have as many “profiles” as the number of datasets in which they appear,
where by \emph{profile} we mean a collection of attributes. Note that each
profile can contain a different subset of attributes (since not all attributes
are present in each dataset) and that a given attribute might take a different
value in different profiles of the same officer (since the iterative matching
procedure is not restricted to performing strict matching).

We are thus faced with the task of “consolidating” the different profiles of
a given officer into a single profile. Of course, if an attribute is present in
a single profile, and absent from the others, this is the value we keep in the
consolidated profile. But if an attribute appears with different values across
different profiles, we choose the value coming from the profile corresponding
to the \emph{most recent data release}.

\paragraph{Cleaning unit assignments.}


\paragraph{Final output.}
