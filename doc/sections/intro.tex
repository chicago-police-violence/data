% !TEX root = ../main.tex


\section{Introduction} \label{sec:intro}


Police brutality inflicts great harm to citizens--families lose their loved ones to fatal shootings, civilians are injured physically and psychologically, and community relationships are disrupted by fear and loss, particularly in communities of color where police misconduct is pervasive. Police misconduct also erodes or destroys any possible foundation of trust between civilians and law enforcement. 

Scholars and policymakers lack the data they need to understand or remedy the problem of police misconduct. Police departments routinely refuse to disclose the relevant data, and other sources of data like media reports are less systematic and reliable [citation to Task Force].Current data collection efforts also face major challenges. For example, owing to political resistance, the Department of Justice (DOJ) only began last year to collect data on arrest-related deaths through the implementation of the ``Death in Custody Reporting Act'' of 2013 (DCRA, P.L. 113-242). Under this act, states are required to provide data regarding the death of any person related to policing activity (detained, under arrest, in the process of being arrested, etc.) But the effectiveness of this program depends on effective presidential support, which was lacking during the Trump Administration. In addition, states often cannot compel local law enforcement to disclose the data in the absence of state legislation. [Center for American Progress, https://www.americanprogress.org/issues/criminal-justice/reports/2021/05/24/499838/address-concerns-data-deaths-custody/  Likewise, the Federal Bureau of Investigation (FBI) ``National Use-of-Force Data Collection'' program,.\footnote{\url{https://crime-data-explorer.app.cloud.gov/pages/home}} is voluntary rather than mandatory. 

Improved data collection and related analysis comes with the promise of improving  police practices and promoting effective police reforms. The 21st Century Task Force, a presidential initiative proposed in 2014 to build trust between citizens and their law enforcement officers has highlighted the need to collect more and better data to ``improve the level of trust, transparency, and accountability between communities and law enforcement agencies.''\footnote{\url{https://cops.usdoj.gov/RIC/Publications/cops-p341-pub.pdf}}. More generally, reliable data enables the wider scientific community to conduct thorough quantitative analysis on police activity and behavior (see, e.g.\ \citet{peeples2019data,peeples2020data} for recent overviews of the problem of the lack of data in the context of police brutality).

In the present work, we provide a new dataset that contains information on the personnel, activities, use of force, and complaints in the Chicago Police Department (CPD).  The original data comprises a collection of datasets, each obtained following different requests covered by the Freedom of Information Act (FOIA) to the Chicago Police Department (CPD) and the Civilian Office of Police Accountability (COPA) on the CPD's personnel and its activities. The data on complaints against police include complaints filed by citizens or internally by other members of the department.  Prior to our work, joint, coherent use of these different dataset for data-analysis purposes was highly non-trivial: for example, publicly available data does not contain coherent unique identifiers for officers across the different datasets. We discuss in detail in \Cref{sec:data} our data cleaning, linking and formatting process. Next, we provide examples and use-cases on the datasets in \Cref{sec:analysis}. These include, but are not limited to, prediction tasks (e.g., predicting officer's misconduct on the basis of their traits, experience, and assigned units), network analysis (e.g., detecting communities within the social network of officers co-listed on complaints), spatiotemporal data analysis (e.g., investigating patterns of officer shooting events), causal inference (e.g., tracking the effects of new disciplinary practices, new training techniques, and new oversight on complaints and use of force). We conclude with a discussion of intended use, and future research directions in \Cref{sec:discussion}.



%Over the last few decades, the problem of police brutality has emerged as a mainstream social and political issue. In the United States,  policymakers have identified the lack of data collection and scarcity of reliable downstream data analysis of policing activities as key elements of this problem. Indeed, improving data collection, and related analysis, comes with the promise of enhancing police, as well as promoting effective police reforms. For example, the 21st Century Task Force, a presidential initiative proposed in 2014 to build trust between citizens and their law enforcement officers, identified the lack of policing data as a major contributor to the inability of communities and law enforcement agencies to ``make informed policy and practice adjustments based on good information''. Relatedly, the task force highlighted the need to collect more and better data, to ``improve the level of trust, transparency, and accountability between communities and law enforcement agencies.''\footnote{\url{https://cops.usdoj.gov/RIC/Publications/cops-p341-pub.pdf}}. 
%
%Small steps in this directions are underway: the Department of Justice (DOJ) is currently collecting data on arrest-related deaths through the implementation of the ``Death in Custody Reporting Act'' of 2013 (DCRA, P.L. 113-242). Under this act, states are required to provide data regarding the death of any person related to policing activity (detained, under arrest, in the process of being arrested, etc.). Starting January 2019, also the Federal Bureau of Investigation(FBI) has started collecting data within the ``National Use-of-Force Data Collection'' program.\footnote{\url{https://crime-data-explorer.app.cloud.gov/pages/home}}. This program is open to all federal, state, local, and tribal law enforcement and investigative agencies, although participation to it is not enforced, rather on voluntary basis. 
%
%However, to the day,  policymakers are fundamentally limited by the insufficient amount, quality and accessibility of policing data. And, more generally, the data-scarcity has prevented the wider scientific community from conducting thorough quantitative analysis on police activity and behavior (see, e.g.\ \citet{peeples2019data,peeples2020data} for recent overviews of the problem of the lack of data in the context of police brutality). As a consequence, we are still far from having developed effective ``data-driven'' policy-making pipelines. Indeed, recent work has investigated how the incompleteness and bias of available data can lead to unwarranted data-driven solutions: \citet{richardson2019dirty} discuss how developing predictive policing system which leverage ``bad'' historical data --- corrupted incomplete and racially biased --- can produce dangerous downstream predictions, potentially violating individual civil rights. Relatedly, machine learning scientists have started investigating the problem of ``algorithmic fairness'', analyzing  risks and flaws related to the employment of black-box, automated decision systems within complex data-driven policy making procedures \citep{veale2018fairness, sloane2019ai, d2020fairness}. Because of all these reasons, it is imperative for the community to collect, curate and provide freely-accessible, easy-to-use data on policing activity.
%
%
%In the present work, we provide a new dataset that contains information on the personnel, activities, use of force, and complaints in the Chicago Police Department (CPD). \textcolor{red}{maybe short discussion of why Chicago is ``special''? + cite some work on policing activity in Chicago? + cite some relevant recent work eg \citep{ba2021role}}. The original data comprises a collection of datasets, each obtained following different requests covered by the Freedom of Information Act (FOIA) to the Chicago Police Department (CPD) and the Civilian Office of Police Accountability (COPA) on the CPD's personnel and its activities. The data on complaints against police include complaints filed by citizens or internally by other members of the department.  Prior to our work, joint, coherent use of these different dataset for data-analysis purposes was highly non-trivial: for example, publicly available data does not contain coherent unique identifiers for officers across the different datasets. We discuss in detail in \Cref{sec:data} our data cleaning, linking and formatting process. Next, we provide examples and use-cases on the datasets in \Cref{sec:analysis}. These include, but are not limited to, prediction tasks (e.g., predicting officer's misconduct on the basis of their traits, experience, and assigned units), network analysis (e.g., detecting communities within the social network of officers co-listed on complaints), spatiotemporal data analysis (e.g., investigating patterns of officer shooting events), causal inference (e.g., tracking the effects of new disciplinary practices, new training techniques, and new oversight on complaints and use of force). We conclude with a discussion of intended use, and future research directions in \Cref{sec:discussion}.