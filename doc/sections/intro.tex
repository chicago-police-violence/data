\section{Introduction}

There has been some ML work done on individual correlates that are associated
with (and potentially predictive of) police misconduct. Scholars have looked at
the association between misconduct and an individual officer's race, gender,
years on the force, prior instances of misconduct, stages of officer career,
assigned unit. But research suggests that violence emerges not just at the
individual level but also at the population level from the interaction of
individuals in groups—think of gangs. 

 
We offer a dataset that allows ML to investigate the influence of police social
networks on police misconduct. The dataset allows ML to explore features of
police social networks that might be associated with and potentially predictive
of adverse police encounters with civilians or other misconduct. These network
correlates can include the architecture and structures of social networks in
departments and smaller subunits within a department—for example, the number of
connections each officer has, the strength of those connections, the bridge
positions that some officers have across subunits. More innovatively, ML can be
used in the way it is often used to study disease dynamics: ML can find the
patterns of behavior (clustering, outbreaks, spreading, geographic hotspots) on
a network that might be associated with violence. 

 
We show how to use complaint datasets from a police department in a major city
to investigate the influence of social networks on police misconduct and
violence. 


The original data was obtained following a series of requests covered by the
Freedom of Information Act (FOIA) to the Chicago Police Department (CPD) and
the Civilian Office of Police Accountability (COPA). The information which was
requested pertained to CPD's personnel and its activities.

\textbf{TODO:} ideally say a bit more about the history of these FOIA requests.
Apparently they were initiated by individual journalists and lawyers and were
later coordinated by the Invisible Institute which ultimately became the
central location were (almost) all the data is currently available.

\begin{table}[h]
	\begin{center}
\begin{tabular}{@{}llll@{}}
	\toprule
	request \#&received&requested&description\\
\midrule
	\texttt{P0-58155}&2017-04-17& &Officer roster\\
	\texttt{P4-41436}&2018-03-21& &Officer roster\\
		\texttt{P0-52262}&2016-12-04&2016-09-19&Unit assignment\\
		\texttt{16-1105}&2016-03-11&2016-02-10&Unit assignment\\
	\texttt{P0-46957}&2016-06-29&2016-04-22&Complaints (CPD)\\
	\texttt{18-060-425}&2018-08-28&2018-08-20&Complaints (COPA)\\
	\texttt{P0-46360}& & &Tactical Response Reports\\
\bottomrule
\end{tabular}
\caption{Summary of the FOIA requests to the CPD and COPA contained in our repository.}
\label{table:summary}
\end{center}
\end{table}

Each FOIA request is identified by a request number, \cref{table:summary} gives
an overview of all the requests made to the CPD and COPA that are present in
our repository. This information is also available in the file
\texttt{dataset.csv} in the root folder of the repository. The original
data—that is the files received after each FOIA request—are present in the
\texttt{raw/} folder of the repository, with one subfolder for each request,
identified by the request number. When available, each subfolder also contains
the formal request letter as well as the reply letter from the CPD or COPA,
which are useful in understanding what data was included in each dataset.
Some additional comments about the data:
\begin{itemize}
	\item \emph{Officer roster:} lists all officers (past or present) employed
		by the CPD along with attributes such as year of birth, age, race,
		gender, appointment date, resignation date, etc.
	\item \emph{Unit assignment:} the CPD is organized into (500 or so?) units.
		Each officer can be assigned to one or multiple units and these
		assignments can change over time. The unit assignment datasets contain
		one record for each officer and each unit they were assigned to, with
		the start date and end date of this assignment.
	\item \emph{Complaints:} formal complaints filed by citizens against police
		officers. Complaints are identified by a complaint number, and there is
		one record for each complaint and each officer listed on the complaint,
		indicating the allegation made against them, result of the
		investigation of the allegation (with possible sanction), etc.
	\item \emph{Tactical Response Reports:} these are forms that officers are
		required to file after each incident for which the officer's response
		involved use of force.
\end{itemize}

Let us already mention an inherent difficulty in making use of this data, which
will be discussed in \cref{sec:linking}: there is no number/identifier which
uniquely identifies officers across datasets. Such an identifier probably
exists internally in the CPD, but was never included in the data released to
the public. One would be tempted to believe that the \emph{badge number} (also
sometimes referred to as \emph{star number} of an officer is such an
identifier, but unfortunately, it changes over the course of an officer's
career in the CPD, and a given badge number can be reassigned to different
officers when they are no longer in use.


\subsection{Related Work}

Police records datasets

Invisible institute repository with the same data
(limitations: not well organized or reproducible, problematic/incorrect linkage, missing unit information, units are not semantically grouped)

Gang violence data

