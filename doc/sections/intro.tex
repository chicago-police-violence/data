% !TEX root = ../main.tex


\section{Introduction} \label{sec:intro}
Accurate and accessible data on policing is crucial to understanding and
remedying the problem of police violence. Police misconduct inflicts harm on
families and communities, particularly those of color in which police presence
is pervasive \cite{Alang17}. In addition to inflicting harm, misconduct also
undermines the foundation of trust between civilians and law enforcement,
particularly when misconduct is not punished \cite{Desmond16}. Accessible data
enables researchers and policymakers to accurately determine causes,
consequences, and potential remedies for police misconduct.
However, data about policing is severely limited, largely because state/local law
enforcement agencies rarely disclose internally collected information
\cite{Jackman21}. Even when departments do disclose information, they often
withhold data linked to individual officers and their professional
relationships.  As a result, scholars and policymakers often
lack the data they need to accurately describe and remedy the problem of police
misconduct.  Rigorous quantitative analysis, which often depends on large
amounts of data pertaining to individual officers, is particularly difficult.

In this work, we provide a dataset containing information on individual
personnel, their activities, shootings, use of force, and complaints filed against them at
the Chicago Police Department (CPD). Chicago is a particularly appropriate
city for a case-study on policing, as it is a major metropolitan area with a
large, diverse police force that has been investigated by the US Department of
Justice due to its high rate of police-involved shootings \cite{DoJ17}. We
describe in detail the dataset contents, as well as 
our procedures for organizing, cleaning and linking the data across multiple
source files in \cref{sec:data}. We provide summary analyses of the
available categories of data in \cref{sec:analysis}.
We finally propose a rich variety of uses for this dataset in \cref{sec:discussion}.

\looseness=-1
The data on individual officers underlying our dataset comes from internal
documents routinely generated by the CPD and disclosed per Freedom of
Information Act requests. The CPD provided this data in thirty-five unlinked,
inconsistent, error-prone, and undocumented spreadsheets. These spreadsheets
were then published online and made available to researchers by the non-profit
Invisible Institute \cite{Kaneya14}. We also draw from data on police shootings
provided by the CPD’s civilian oversight board. Prior to our work, linking
information across the various heterogeneous data sources was a serious
challenge: an officer described in one source is not uniquely identifiable as
the same officer in other sources due to inconsistent fields, missing data, and
time-varying attributes.  The value of our work thus lies in (1) consolidating
these heterogeneous files to create a clean, well-organized, and
well-integrated dataset on individual officers, (2) providing code that
reproducibly builds the dataset from the raw source files, and (3) a detailed
description of the method we developed to carefully link data files.  
