% !TEX root = ../main.tex


\section{Introduction} \label{sec:intro}

Data about policing is severely limited, largely because law enforcement
agencies at the state and local level do not ordinarily disclose information
collected by the departments. Even when departments do disclose information,
they often withhold data linked to individual identifiable officers or
information about the officers’ relationship to each other. As a result,
scholars and policymakers who study policing often lack the data they need to
accurately describe and remedy the problem of police misconduct. Rigorous
quantitative analysis, which often depends on massive amounts of data on
individual officers, is particularly difficult.

Accurate and accessible data on policing is crucial to understanding and
remedying the problem of police violence. Police misconduct can inflict deadly
harm on families and communities, particularly those communities of color in
which police presence is pervasive.  In addition to inflicting physical and
psychological harm on civilians, police misconduct can also undermine the
foundation of trust between civilians and law enforcement, particularly when
misconduct is not punished. Accessible data enables researchers and
policymakers to more accurately determine causes, consequences and potential
remedies for police misconduct.

We provide a cleaned, well-integrated dataset that contains information on
individual personnel, their activities, use of force, and complaints filed
against them at the Chicago Police Department (CPD). We describe in detail the
dataset contents and provide summary statistics on the available categories of
data. We also provide code and descriptions detailing our procedures for
organizing, cleaning and linking the data across multiple spreadsheets and
sources. We propose a rich variety of uses for this dataset, to include some
machine learning approaches to analysis.

The data on individual officers underlying our dataset comes from internal
documents routinely generated by the CPD and disclosed under court order in
a Freedom of Information Act lawsuit. The CPD provided this data in thirty-five
unlinked, inconsistent and undocumented spreadsheets. These spreadsheets were
then published online and made available to researchers by a non-profit
organization, the Invisible Institute. We also draw from data on
police-involved shootings provided by the CPD’s civilian oversight board.

The value of our work lies primarily in creating a well-documented,
well-integrated dataset on individual officers and documenting our data
cleaning, linking and formatting process, to make the process easy to
reproduce. In particular, we describe the method we used to carefully link
a wide range of data to create a well-integrated set of information on
individual officers. Prior to our work, linking information across data sources
and spreadsheets was difficult because an officer described in one spreadsheet
was not uniquely identifiable as the same officer in other spreadsheets. We
describe and document a process to uniquely identify officers across
spreadsheets, making integration much easier.

Last but not least, we review a rich variety of potential uses for this
integrated dataset. These include, but are not limited to, machine-learning
style prediction tasks (e.g., predicting officer's misconduct on the basis of
their traits, experience, and assigned units), network structure analysis
(e.g., detecting communities within the social network of officers co-listed on
complaints), spatiotemporal data analysis (e.g., investigating dynamic patterns
of officer shooting events), causal inference (e.g., tracking the effects of
new disciplinary practices, new training techniques, and new oversight on
complaints and use of force). We conclude with a discussion of intended use,
and future research directions in \Cref{sec:discussion}.

[Chicago is a particularly appropriate city for a case-study on policing.
Chicago is a major metropolitan area with a large, diverse police force. In
addition, the Chicago Police Department has been investigated by the US
Department of Justice because of the Department’s very high rate of
police-involved shootings. Finally, the FOIA data available from Chicago is
particularly fine-grained, and contains key information not available from
other departments on officer identity; office information permits integration
of a wide-range of data across a range of sources on individual officers.]

\iffalse
Police brutality inflicts great harm to citizens--families lose their loved
ones to fatal shootings, civilians are injured physically and psychologically,
and community relationships are disrupted by fear and loss, particularly in
communities of color where police misconduct is pervasive. Police misconduct
also erodes or destroys any possible foundation of trust between civilians and
law enforcement. 

Scholars and policymakers lack the data they need to understand or remedy the
problem of police misconduct. Police departments routinely refuse to disclose
the relevant data, and other sources of data like media reports are less
systematic and reliable [citation to Task Force].Current data collection
efforts also face major challenges. For example, owing to political resistance,
the Department of Justice (DOJ) only began last year to collect data on
arrest-related deaths through the implementation of the ``Death in Custody
Reporting Act'' of 2013 (DCRA, P.L. 113-242). Under this act, states are
required to provide data regarding the death of any person related to policing
activity (detained, under arrest, in the process of being arrested, etc.) But
the effectiveness of this program depends on effective presidential support,
which was lacking during the Trump Administration. In addition, states often
cannot compel local law enforcement to disclose the data in the absence of
state legislation. [Center for American Progress,
https://www.americanprogress.org/issues/criminal-justice/reports/2021/05/24/499838/address-concerns-data-deaths-custody/
Likewise, the Federal Bureau of Investigation (FBI) ``National Use-of-Force
Data Collection''
program,.\footnote{\url{https://crime-data-explorer.app.cloud.gov/pages/home}}
is voluntary rather than mandatory. 

Improved data collection and related analysis comes with the promise of
improving  police practices and promoting effective police reforms. The 21st
Century Task Force, a presidential initiative proposed in 2014 to build trust
between citizens and their law enforcement officers has highlighted the need to
collect more and better data to ``improve the level of trust, transparency, and
accountability between communities and law enforcement
agencies.''\footnote{\url{https://cops.usdoj.gov/RIC/Publications/cops-p341-pub.pdf}}.
More generally, reliable data enables the wider scientific community to conduct
thorough quantitative analysis on police activity and behavior 
(see, e.g.~\citet{peeples2019data,peeples2020data} for recent overviews of the problem of
the lack of data in the context of police brutality).

In the present work, we provide a new dataset that contains information on the
personnel, activities, use of force, and complaints in the Chicago Police
Department (CPD).  The original data comprises a collection of datasets, each
obtained following different requests covered by the Freedom of Information Act
(FOIA) to the Chicago Police Department (CPD) and the Civilian Office of Police
Accountability (COPA) on the CPD's personnel and its activities. The data on
complaints against police include complaints filed by citizens or internally by
other members of the department.  Prior to our work, joint, coherent use of
these different dataset for data-analysis purposes was highly non-trivial: for
example, publicly available data does not contain coherent unique identifiers
for officers across the different datasets. We discuss in detail in
\Cref{sec:data} our data cleaning, linking and formatting process. Next, we
provide examples and use-cases on the datasets in \Cref{sec:analysis}. These
include, but are not limited to, prediction tasks (e.g., predicting officer's
misconduct on the basis of their traits, experience, and assigned units),
network analysis (e.g., detecting communities within the social network of
officers co-listed on complaints), spatiotemporal data analysis (e.g.,
investigating patterns of officer shooting events), causal inference (e.g.,
tracking the effects of new disciplinary practices, new training techniques,
and new oversight on complaints and use of force). We conclude with a
discussion of intended use, and future research directions in
\Cref{sec:discussion}.
\fi



%Over the last few decades, the problem of police brutality has emerged as a
%mainstream social and political issue. In the United States,  policymakers
%have identified the lack of data collection and scarcity of reliable
%downstream data analysis of policing activities as key elements of this
%problem. Indeed, improving data collection, and related analysis, comes with
%the promise of enhancing police, as well as promoting effective police
%reforms. For example, the 21st Century Task Force, a presidential initiative
%proposed in 2014 to build trust between citizens and their law enforcement
%officers, identified the lack of policing data as a major contributor to the
%inability of communities and law enforcement agencies to ``make informed
%policy and practice adjustments based on good information''. Relatedly, the
%task force highlighted the need to collect more and better data, to ``improve
%the level of trust, transparency, and accountability between communities and
%law enforcement
%agencies.''\footnote{\url{https://cops.usdoj.gov/RIC/Publications/cops-p341-pub.pdf}}. 
%
%Small steps in this directions are underway: the Department of Justice (DOJ)
%is currently collecting data on arrest-related deaths through the
%implementation of the ``Death in Custody Reporting Act'' of 2013 (DCRA, P.L.
%113-242). Under this act, states are required to provide data regarding the
%death of any person related to policing activity (detained, under arrest, in
%the process of being arrested, etc.). Starting January 2019, also the Federal
%Bureau of Investigation(FBI) has started collecting data within the ``National
%Use-of-Force Data Collection''
%program.\footnote{\url{https://crime-data-explorer.app.cloud.gov/pages/home}}.
%This program is open to all federal, state, local, and tribal law enforcement
%and investigative agencies, although participation to it is not enforced,
%rather on voluntary basis. 
%
%However, to the day,  policymakers are fundamentally limited by the
%insufficient amount, quality and accessibility of policing data. And, more
%generally, the data-scarcity has prevented the wider scientific community from
%conducting thorough quantitative analysis on police activity and behavior
%(see, e.g.\ \citet{peeples2019data,peeples2020data} for recent overviews of
%the problem of the lack of data in the context of police brutality). As a
%consequence, we are still far from having developed effective ``data-driven''
%policy-making pipelines. Indeed, recent work has investigated how the
%incompleteness and bias of available data can lead to unwarranted data-driven
%solutions: \citet{richardson2019dirty} discuss how developing predictive
%policing system which leverage ``bad'' historical data --- corrupted
%incomplete and racially biased --- can produce dangerous downstream
%predictions, potentially violating individual civil rights. Relatedly, machine
%learning scientists have started investigating the problem of ``algorithmic
%fairness'', analyzing  risks and flaws related to the employment of black-box,
%automated decision systems within complex data-driven policy making procedures
%\citep{veale2018fairness, sloane2019ai, d2020fairness}. Because of all these
%reasons, it is imperative for the community to collect, curate and provide
%freely-accessible, easy-to-use data on policing activity.
%
%
%In the present work, we provide a new dataset that contains information on the
%personnel, activities, use of force, and complaints in the Chicago Police
%Department (CPD). \textcolor{red}{maybe short discussion of why Chicago is
%``special''? + cite some work on policing activity in Chicago? + cite some
%relevant recent work eg \citep{ba2021role}}. The original data comprises a
%collection of datasets, each obtained following different requests covered by
%the Freedom of Information Act (FOIA) to the Chicago Police Department (CPD)
%and the Civilian Office of Police Accountability (COPA) on the CPD's personnel
%and its activities. The data on complaints against police include complaints
%filed by citizens or internally by other members of the department.  Prior to
%our work, joint, coherent use of these different dataset for data-analysis
%purposes was highly non-trivial: for example, publicly available data does not
%contain coherent unique identifiers for officers across the different
%datasets. We discuss in detail in \Cref{sec:data} our data cleaning, linking
%and formatting process. Next, we provide examples and use-cases on the
%datasets in \Cref{sec:analysis}. These include, but are not limited to,
%prediction tasks (e.g., predicting officer's misconduct on the basis of their
%traits, experience, and assigned units), network analysis (e.g., detecting
%communities within the social network of officers co-listed on complaints),
%spatiotemporal data analysis (e.g., investigating patterns of officer shooting
%events), causal inference (e.g., tracking the effects of new disciplinary
%practices, new training techniques, and new oversight on complaints and use of
%force). We conclude with a discussion of intended use, and future research
%directions in \Cref{sec:discussion}.
