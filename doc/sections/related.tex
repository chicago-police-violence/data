\section{Related Work}\label{sec:related}
Existing public datasets on policing are extremely limited; 
few agencies disclose internal personnel data to the public \cite{Jackman21},
and other data collected by police (e.g., to build ``early intervention'' warning systems)
remain confidential, even when scholars publish the results of their 
work on such systems \cite{Helsby18}.
In response to public pressure, however, a number of
public datasets on policing have recently emerged: 
the Police Data Initiative, a collection of datasets from 
more than 130 state and local law enforcement agencies \cite{pdi};
Stanford University’s Open Policing Initiative, which contains
data on traffic stops from around the country \cite{sopp}; and
the NYPD Misconduct Complaint Database,
which includes complaints regarding officers in the New York Police Department \cite{nmcd}.
However, these datasets do not include information about individual officer behavior,
contain only raw, inaccessible data, or both.
Furthermore, the Department of Justice
(DOJ) only began last year to collect data on arrest-related deaths through the
implementation of the ``Death in Custody Reporting Act'' of 2013\footnote{Efforts to implement and enforce the law have been delayed for several years. According to a 2018 report by the DOJ’s Office of the Inspector General, this delay was due in part to confusion over the law’s requirements and the failure of the DOJ to agree on a proposal for data collection.} \cite{DICRA2013}. Under this act, states are required to provide data regarding
the death of any person related to policing activity. But states often cannot compel local
law enforcement to disclose the data in the absence of state legislation.
Likewise, the Federal Bureau of Investigation (FBI) ``National Use-of-Force
Data Collection'' program is voluntary \cite{Gardner2020}.

Other work has drawn on the same raw data that we clean and integrate in this
work. The Invisible Institute itself released a processed version of
the data that has been used in previous work \cite{invisdata,ba2021role}. However, the
code underlying this release suffers from gaps in its methodology
and documentation, which severely limits the usability and reproducibility of
the data. The raw data has also been used directly in concert with litigation
settlements involving CPD officers \cite{Rozema19}, but no public information
is available on how the raw data were cleaned or linked.
