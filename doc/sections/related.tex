\section{Related Work}\label{sec:related}

Existing public datasets on policing are extremely limited. All police
departments collect data on police personnel, activities, operations and
complaints, but few agencies disclose such data to the public \cite{Jackman21}.
A small but increasing number of departments are using the data to develop
machine learning approaches to “early intervention” warning systems that
predict adverse encounters between civilians and police. However, the datasets
used for these “early intervention systems” remain confidential, even when
scholars publish the results of their work on such systems.

In response to public pressure on the issue of police violence, a number of
public datasets on policing have recently emerged. These datasets, which are
generated by both private and public actors, are hard to reproduce and often
incomplete. For example, the Police Data Initiative (sponsored by a non-profit
on science in policing) is a collection of datasets from more than 130 state
and local law enforcement. These datasets often contain raw, uncleaned data,
and almost never include information about individual officer behavior.
Stanford University’s Open Policing Initiative, which contains data on traffic
stops from around the country, does not include information about identifiable
individual officers.

The NYPD Misconduct Complaint Database, which has been compiled by the New York
Civil Liberties Union, includes complaint data made available by the New York
Police Dept. Civilian Complaint Review Board (a civilian oversight
organization.) This dataset includes information identifying officers by name
and is searchable as well, but the dataset includes only uncleaned data.
 
Current data collection efforts by government agencies can overcome some of
these limits, but can also face political and legal challenges. Owing to
political resistance by the Trump Administration, the Department of Justice
(DOJ) only began last year to collect data on arrest-related deaths through the
implementation of the ``Death in Custody Reporting Act'' of 2013. Under this
act, states are required to provide data regarding the death of any person
related to policing activity (detained, under arrest, in the process of being
arrested, etc.) But states often cannot compel local law enforcement to
disclose the data in the absence of state legislation. Likewise, the Federal
Bureau of Investigation (FBI) ``National Use-of-Force Data Collection'' program
is voluntary and not mandatory.

Policing scholars have compiled their own datasets. In a paper on the impact of
race and gender on policing, economist Bocar Ba and his research team generated
a publicly available dataset with information on individual officers in the
Chicago Police Department. This dataset linked officer demographic information
with their arrests, stops, assignment history, and data on the districts to
which the officers were assigned. Documentation for this dataset was
appropriate but sparse, making the reproduction of the dataset from scratch
somewhat challenging.

Other scholars have drawn on the same Invisible Institute data to create
a dataset similar to ours, but with far less documentation and focus on
reproducibility. In a paper on the impact of complaints on misconduct, Legal
scholars Rozema and Schanzenbach used the Invisible Institute raw data files to
link information on officer complaints with litigation settlements involving
the officers. However, neither the paper nor the publicly available dataset
provided any information about cleaning, processing or linking the data. 
