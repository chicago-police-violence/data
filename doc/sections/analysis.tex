% !TEX root = ../main.tex

\section{Exploratory Analysis} \label{sec:analysis}

In this section, we present an exploratory analysis of the different entities
present in the cleaned and linked data, with the purpose of providing some
insight into the data as well as potential pitfalls in its use. Code for these
analyses is available in the \texttt{examples/} folder of the repository.  Note
that throughout this section, officer race and gender are binned per the CPD's
coarse categories.  Unless otherwise specified, an officer is considered
``active'' if their resignation date is after 2019-01-01. 

\paragraph{Roster and Units.} \cref{fig:history,fig:units,tab:stats} 
provide summaries of demographics, age, and unit assignments
of the roughly 35\,000 officers present in the data,
whose appointment dates range from 1936 to 2018.
\cref{fig:history} in particular highlights an important limitation of the data:
although there appears to be a steep increase in the number of active officers
until the 1980s, it is much more likely that a significant fraction of
officers is missing from the database during those early years.
Since the process through which officers were added to these records is unclear,
the roster data should be assumed to contain only a subset of officers prior to 
the 1980s. Another point of interest---demonstrated in \cref{fig:units}---is that officers
most commonly join precisely 2 units in their career: Unit 44 (the training academy),
and their sole assignment.
 
\begin{table}[t!]
\caption{Counts of officers (first row) and active officers (second row).} \label{tab:stats}
\begin{tabular}{l|c|c|c|c|c|c|c|c|c|}
\cline{2-3} \cline{5-10}
                                               & \multicolumn{2}{c|}{\textbf{Gender}} & \multicolumn{1}{l|}{} & \multicolumn{6}{c|}{\textbf{Race}}                                                                                                                                                   \\ \cline{2-3} \cline{5-10} 
                                               & {\textbf{M}}   & {\textbf{F}}   &                       & {\textbf{White}} & {\textbf{Black}} & \multicolumn{1}{l|}{{\textbf{Hisp.}}} & {\textbf{Asian/P.I.}} & \multicolumn{1}{l|}{{\textbf{Indig.}}} & {\textbf{Bl. Hisp.}} \\ \cline{1-3} \cline{5-10} 
\multicolumn{1}{|c|}{\textbf{All}}    & 28316                 & 7122                  &                       & 21047                   & 8599                    & 4811                                         & 582                     & 67                                              & 9                           \\ \cline{1-3} \cline{5-10} 
\multicolumn{1}{|c|}{\textbf{Active}} & 11118                 & 4452                  &                       & 7241                    & 3895                    & 3596                                         & 467                     & 40                                              & 9                           \\ \cline{1-3} \cline{5-10} 
\end{tabular} 
\end{table}

\begin{figure}[t!] 
\includegraphics[width=\textwidth]{figs/history} 
\includegraphics[width=\textwidth]{figs/history_by} 
\caption{Officer birth years (bottom), appointments (top left), resignations (top center), and active
officers (top right) appearing in the CPD roster database from the years 1940 to 2019.}
\label{fig:history}
\end{figure}


\begin{figure}[t!] 
	\includegraphics[width=\textwidth]{figs/units_officers.pdf} 
	\caption{Histograms of the number of unit assignments for
officers over their career (left), and the number of years spent in each unit for assignments that had terminated by 2019-01-01 (right).}
\label{fig:units}
\end{figure}

%\begin{figure}[t!] 
%\begin{subfigure}{0.44\textwidth}
%	\includegraphics[width=\textwidth, clip, trim= 970 0 0 0]{figs/complaints_times} 
%\end{subfigure}
%\begin{subfigure}{0.52\textwidth}
%	\includegraphics[width=\textwidth]{figs/complaints} 
%\end{subfigure}
%	\caption{Complaints filed by year (left) and complaints per officer by race and gender (right).} \label{fig:complaints}
%\end{figure}

\begin{figure}[t!] 
	\includegraphics[width=\textwidth]{figs/complaints_years_race_gender.pdf} 
	\caption{Complaints filed by year (left) and complaints per officer by race and gender (right).} \label{fig:complaints}
\end{figure}

\begin{figure}[t!] 
%\begin{subfigure}{0.5\textwidth}
%\raisebox{.5cm}{
%\includegraphics[width=\textwidth]{figs/salary} 
%}
%\end{subfigure}
%\begin{subfigure}{0.5\textwidth}
%\includegraphics[width=\textwidth]{figs/awards} 
%\end{subfigure}
\includegraphics[width=\textwidth]{figs/salary_awards.pdf} 
\caption{Officer salaries as a function of the number of years in a selection of representative positions (left),
and the number of award requests per officer by race and gender (right).} \label{fig:salary_awards}
\end{figure}
\begin{figure}[t!] 
	\includegraphics[width=\textwidth]{figs/trr_stats} 
	\caption{Summary counts of TRRs by race, gender, and injury status.} \label{fig:trrs_stats1}
\end{figure}


\paragraph{Complaints.} 
We report patterns of complaints as functions of both time and officer
demographics in \cref{fig:complaints}. This figure highlights another important
feature of the data: the number of complaints filed gradually reduces over the
years, potentially as a consequence of the perception of ineffectiveness of
such complaints \cite{xx}.  It is also clear in this figure that male officers
of color receive proportionally more complaints than both female officers and
white officers, indicating potential racial bias in complaint filings.
Supplemental figures in \cref{sec:additional_figs} further demonstrates that
fewer complaints are filed during the weekends and colder months.

\paragraph{Salary and Awards.}
\looseness=-1
\cref{fig:salary_awards} shows the officer salary for a selection of
representative positions versus years of experience, as well as the number of
award requests filed for officers by demographic group.  Unsurprisingly, salary
generally increases with experience and rank, and \cref{fig:salary_gender_race}
in the appendix shows that salary has only a marginal dependency on race and
gender, likely because officers are unionized with strict rules about salary
progression. However, \cref{fig:salary_awards} and \cref{fig:position} both
hint at implicit forms of gender bias present in the data. In particular,
awards appear to be requested at a higher rate for male officers than for
female officers, and similarly male officers disproportionally occupy higher
ranks than their female counterparts. \cref{fig:position} also suggests that
white officers are more readily promoted through the lower ranks (officer,
detective, sergeant) than officers of color.  This trend seem to disappear for
higher ranks (lieutenant and above), although drawing firm conclusions would
require a rigorous statistical analysis given the much smaller number of
officers occupying these ranks.

\paragraph{Tactical Response Reports.}
\cref{fig:trrs_stats1} presents summaries of the tactical response reports in
the data.  In most cases, neither the officers nor the subjects involved are
injured, although civilians get injured at a much higher rate (left subplot),
and officers tend to fire first (4593 instances versus 329).  Analysis
involving these data should account for racial bias in officers' use of force:
black civilians are the subject of a TRR over 4 times more often than any other
race. \Cref{fig:trrs_times} in the appendix also provides some insight into the
temporal nature of TRRs: they are filed more frequently at night, on the
weekends, and during warmer months. There is an unusual spike in TRRs between
2010 and 2012 that warrants further investigation.


%\begin{tabular}{c|c|c|c|}
%\cline{2-4}
%                                                    & \textit{Member} & \textit{Subject} & \textit{Other} \\ \hline
%\multicolumn{1}{|c|}{\textit{\textbf{Fired first}}} & 4593            & 171               & 158            \\ \hline
%\end{tabular}



